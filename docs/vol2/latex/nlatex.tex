\documentstyle[vol2,twoside]{attart}
% \XX{263}{LaTeX User Guide}{latex}
\XX{263}{Latex User Guide}{latex}

\batchmode
\textwidth 6.5in
% don't lower X by .7em.  use .5em like \TeX
\def\TeX{Tex}
% \def\LaTeX{{\rm L\kern-.36em\raise.3ex\hbox{\sc a}\kern-.15em
%     T\kern-.1667em\lower.5ex\hbox{E}\kern-.125emX}}
\def\LaTeX{Latex}
% \def\BibTeX{{\rm B\kern-.05em{\sc i\kern-.025em b}\kern-.08em
%     T\kern-.1667em\lower.5ex\hbox{E}\kern-.125emX}}
\def\BibTeX{Bibtex}
\def\UNIX{UNIX}
\hyphenchar\tentt=-1 % don't hyphen the literal terms
\def\@biblabel#1{#1.\hfill}

% change format of \index output
\makeatletter
\def\@wrindex#1#2{\let\thepage\relax
   \xdef\@gtempa{\write#1{%\string
      #2<tab>\thepage}}\endgroup\@gtempa
   \if@nobreak \ifvmode\nobreak\fi\fi\@esphack}
\makeatother
\makeindex

\pagestyle{headings}
% title is set by \XX second argument
\author{Howard Trickey}

\abstracttext{\LaTeX\ is a standard set of macros for \TeX.
The most important feature of \LaTeX\ is that the user is
not responsible for details of spacing, font selection,
and page layout---instead, various {\em styles\/} are provided
and the user can print the same document in any desired style.
\BibTeX\ is a program which performs a similar
style-based formatting service for bibliographic references.

The basic description of \LaTeX{}
leaves certain things to be described by a {\em Local Guide}.
This paper serves as the {\em Local Guide\/} for our \UNIX\regmark\ system.
It describes the commands needed to run \LaTeX\ and \BibTeX,
as well as locally available document and bibliography styles.
}

\begin{document}

\maketitle

\section{Introduction}

You should read the \LaTeX\ book\cite{latex} to learn how to
prepare \LaTeX\ documents.
This paper describes the \UNIX\ commands related to \LaTeX,
and additional available document and bibliography styles.

\LaTeX\ is invoked as
\begin{eg}
	latex paper
\end{eg}
where \verb|paper.tex| is the filename of your document
(it must end in \verb|.tex|).
If all goes well, \LaTeX\ will process all of the file
and produce \verb|paper.dvi|, an output file which can be
printed by
\begin{eg}
	lp paper.dvi
\end{eg}
or viewed on a graphics terminal by
\begin{eg}
	dvit paper.dvi > paper.t
	proof paper.t
\end{eg}
(Use the {\em dvit\/} flag \verb|-Tgnot| or \verb|-Tjerq|, depending
on where {\em proof\/} is run.)
You might want to try
\begin{eg}
	latex sample
	lp sample.dvi
\end{eg}
\LaTeX\ searches the standard \TeX\ macro directory%
\footnote{Usually \verb|/usr/lib/tex/macros|.}
if it can't find a file in your current directory,
and Lamport has supplied a \verb|sample.tex| there.

If there are errors in the document, \LaTeX\ will print a message
on the terminal and expect you to interact with it as described
in Section~2.3 of the \LaTeX\ book.  The simplest thing to do
is type \verb|x| to exit.

There are longer documents describing many of the things summarized
in this paper.  They are kept in the ``documentation directory'',
which is {\tt doc} in the standard \TeX\ macro directory (but only
on some machines).

\section{Document Styles}

A \verb|\documentstyle| command at the beginning of a \LaTeX\ document
specifies the formatting style to be used.
A document command like {\tt \verb|\|section$\{$Document Styles$\}$} has to be
turned into a sequence of spacing and font changing commands, and
a {\em style designer\/} has encoded these choices into {\em style files}.%
\footnote{Style files have a \verb|.sty| extension and can be found
in the \TeX\ macro directory.}
The \LaTeX\ book describes the standard styles {\tt article},
{\tt letter}, {\tt report}, and {\tt slides}.
This section describes some other available document styles.

Ideally, you should be able to change only the argument to
\verb|documentstyle| to get a new style.
That is mostly true, but there are usually minor differences---to
supply information that is only relevant to certain styles (such as
a Technical Memorandum Number), and to accommodate occasional
laziness on the part of style file writers (they will ask you to
list the authors and their affiliations in a certain way).

\subsection{The {\tt tm} Style}
\label{tmstylesec}

AT\&T Bell Laboratories technical memoranda can be prepared using
the {\tt tm} style.
It looks like the {\tt article} style, but makes a title page
and cover sheet in the AT\&T technical memo style.
Your input file should look like:
\begin{eg}
\documentstyle{tm}
\title{A Sample Document}
\author{Howard Trickey}
\initials{HWT}
\eaddress{research!howard}
\department{11273}
\location{MH}{2C460}{6831}
\documentno{890814}{01}{TM}
\date{\today}
\chargecase{311403-0101}
\filecase{39199-11}
\keywords{typesetting; TeX}
\mercurycode{CMP}
\memotype{TECHNICAL MEMORANDUM}
\abstract{...}
\begin{document}
\makehead
\makeabstract
\section{Introduction}
...
\makesignature

\noindent ATTs.\\
Figures\\
References

\medskip
\copyto{All Supervision Center 1127\\
H. Green}
\coverto{All Members of Department 1127\\
J. Smith}

\begin{thebibliography}{99}
...
\end{thebibliography}
\coversheet
\end{document}
\end{eg}
The \verb|\author|, \verb|\initials|, \verb|\department|, and
\verb|\location|%
\footnote{The third argument to {\tt $\backslash$location} is the phone extension.
All three arguments can be blank for outside authors.}
are all required.  If there are other authors,
supply information for them after the first, starting each
with \verb|\author|.
For each author, one can add any of:
\begin{eg}
\locationextra{...}
\company{Some Other Co.}
\maddress{12 Main St.\\
    Anyville, CA 97999}
\end{eg}
The other commands in the above example are self-explanatory.
The items \verb|\chargecase|, \verb|\filecase|, and \verb|\mercurycode|
can be repeated.

A number of other commands can be used with the {\tt tm} style:

\begin{itemize}
\item[] \hbox{\verb|\copytocov{|{\em name}\verb|\\|...\verb|}|}
	like \verb|\copyto| but listing just on the cover sheet,
	not the point of invocation too.
\item[] \hbox{\verb|\markdraft|}
	put {\em DRAFT\/} at the bottom of pages.
\item[] \hbox{\verb|\markprivate|}
	put {\bf PRIVATE} at the bottom of pages.
	Also, there are \verb|\markproprietary|
	and \verb|\markrestricted| commands.
\item[] \hbox{\verb|\classified{|{\em classification level}\verb|}|}
	government assigned title (Secret, Top Secret, etc.)
\item[] \hbox{\verb|\itdsrestrict|}
	tell ITDS to obtain approval prior to releasing to
	AT\&T employees.
\item[] \hbox{\verb|\UNIX|}
	UNIX\regmark, with a footnote at the first usage.
\item[] \hbox{\verb|\tm|,\verb|\regmark|}
	trademark `\tm' and registered trademark `\regmark'.
\item[] \hbox{\verb|\markedfootnote{|{\em symbol}\verb|}{|{\em text}\verb|}|}
	footnotes using the given symbol as footnote mark.
\item[] \hbox{\verb|\appendices|}
	use to begin having sections being labeled by A, B, C, etc.
\end{itemize}

Consult \verb|tm.tex| in the documentation directory for more
information and an example of the {\tt tm} style.

\subsection{The {\tt rp} Style}

The {\tt rp} style takes input in the same format as the {\tt tm}
style, though many of the commands are unnecessary, and produces
a ``released paper'' style.
You have to provide a city using
\begin{eg}
\cityaddr{Murray Hill, NJ 07974}
\end{eg}

\subsection{The {\tt attart} Style}

This is an approximation to the {\tt troff\/} \verb|-ms| style.
The headings are smaller and the margins narrower than those of the
\verb|article| style.
You should use one of the following style file options:
\begin{itemize}
\item[{\tt tm}\hfill] Technical Memo (without coversheet).
\item[{\tt rp}\hfill] Released Paper (separate title page).
\item[{\tt vol2}\hfill] ``Volume 2'': the two-column style you are reading now.
\end{itemize}

The input file should look like input for the \verb|article|
style (as for example, in Figure C.3 of the \LaTeX\ book),
except that there should be
\begin{eg}
\abstracttext{...abstract...}
\end{eg}
before the \verb|\maketitle| command.
As with the \verb|article| style, multiple authors
should be given in a single \verb|\author| command,
separated by \verb|\and|'s.  Author affiliation is put
either after each author (using \verb|\\| to get line breaks),
or with the footnoting \verb|\thanks|.

The {\tt tm} option isn't as good as the {\tt tm} style of
section~\ref{tmstylesec}, but you might prefer the formatting.
To use it, the following has to be provided in the preamble:
\begin{eg}
\wpnumber{311403-0101}
\filenumber{39199-11}
\tmnumber{11273-851009-11TM}
\end{eg}
and the signature block should be created via
\begin{eg}
\sigblock{MH-11273/HWT}{10}{11}
\end{eg}
(The second argument is the number of references, the third is
the number of figures.)

\subsection{The {\tt siam} Style}

The {\tt siam} style looks like the one used by the
{\em Society for Industrial and Applied Mathematics (SIAM)\/)}
in its publications.
It is a compact style, with indented run-in section headings
and small type sizes.

Prepare your document as for the \verb|article| style.
After the \verb|\maketitle|, the paper should continue:
\begin{eg}
\begin{abstract}...\end{abstract}
\begin{keywords}...\end{keywords}
\begin{AMSMOS}...\end{AMSMOS}
\end{eg}
followed by the paper body.
Other things to note:
\begin{itemize}
\item  Give author affiliations via \verb|\thanks| footnotes in \verb|\author|.
\item  Place table \verb|\caption|'s above the table material.
Place figure \verb|\caption|'s below the figures.
\end{itemize}

The documentation directory contains an example, \verb|siam.tex|.
A larger example---a \LaTeX\ introduction and hints for
mathematics typing---is \verb|p.tex| in the numerical analysis
\TeX\ directory.\footnote{\verb|/usr/num/tex| on many machines.}

\subsection{The {\tt acmtrans} Style}

The {\tt acmtrans} style can be used to produce documents
for the ACM transactions.  Sans serif fonts are used for
the title and section headings.

Input files are like those for \verb|article|, with minor changes.
The \verb|\author| command has to have its information in
a special form
\begin{eg}
\author{JAMES E. ARCHER, JR.
       \\ Rational Machines
    \and RICHARD CONWAY and
         FRED B. SCHNEIDER\\
         Cornell University}
\end{eg}
with names in all uppercase, and authors with the same
affiliation separated by ``\verb|and|'' instead of ``\verb|\and|''.
The following material has to come before \verb|\maketitle|:
\begin{eg}
\begin{abstract}...\end{abstract}
\category{I.7.2}%
    {Text Processing}%
    {Document Preparation}%
    [Languages \and
     Photocomposition]
\terms{Documentation, Languages}
\keywords{...}
\end{eg}
The optional argument of \verb|\category| contains subject descriptors,
with \verb|\and| separating multiple descriptors.

In the document body
\begin{itemize}
\item
Use a \verb|longenum| and \verb|longitem| environments instead
of \verb|enumerate| and \verb|itemize| when
the items are more than two or three lines long.
\item
To itemize with a \verb|describe| list, with a mandatory argument giving the
longest item, to itemize with user-supplied labels.
\item
Environments for things with proofs, such as theorems and lemmas,
should be defined with the \verb|newtheorem| command.
A \verb|proof| environment can follow these.
Use \verb|newdef| (like \verb|newtheorem|, with no optional arguments)
for things like definitions and examples.  You can suppress automatic
number of both kinds by supplying an optional argument to replace
the number.
\item
Computer programs can be formatted with the \verb|program| environment,
which is like \verb|tabbing|, but with a smaller font size.
Start flush left, and indent by 1 em at a time.
\item
Figures go in a \verb|figure| or \verb|narrowfig| environment,
depending on their width.
Tables go in an \verb|acmtable| environment.
Figures and Tables without captions should have \verb|\nocaption|
instead of \verb|\caption|.
\end{itemize}

The documentation directory contains \verb|acmtr.tex|, an example
a bit more information about this style.

\subsection{The {\tt cup} Style}

The {\tt cup} style is an attempt at the Cambridge University Press
house style.
Only an 11 point version is available at this time.
In addition to the usual \verb|\author| and \verb|\title|,
you should supply your affiliation as
\begin{eg}
\affiliation{AT\hskip-.6pt\&T
             Bell Laboratories}
\end{eg}


\subsection{The {\tt attletter} Style}

The {\tt attletter} style is like the standard {\tt letter} style,
except that it puts an AT\&T letterhead at the top.
Personal information for the letterhead can be given as:
\begin{eg}
\signature{Howard Trickey}
\streetaddr{600 Mountain Avenue}
\usaaddr{USA}
\location{Room 2C460}
\telephone{(201) 582-6831}
\arpanet{howard%
     {\egtrm @}research.att.com}
\uucp{research!howard}
\end{eg}
Only \verb|\signature| is required.
The street address defaults to the one shown;
there are no other defaults.
If you prefer to feed preprinted letterhead paper into
the printer, use the \verb|\address| command to define
the information to appear below the preprinted logo.

You can get a ``{\tt Ref:}'' line before the opening by putting
this in the preamble:
\begin{eg}
\ref{...}
\end{eg}

\subsection{Document Style Options}

Section C.4.1 in the \LaTeX\ book describes the style options
\verb|11pt|, \verb|12pt|, \verb|twoside|, \verb|twocolumn|,
\verb|titlepage|, \verb|openbib|, 
\verb|leqno|, and \verb|fleqn|.
Here are some others.
They work with any style, unless otherwise noted.

\begin{itemize}
\item[] \hbox{\tt alltt}
Defines an {\tt alltt} environment which is like {\tt verbatim}
except that `\verb|\|', `\verb|\{|', and `\verb|\}|' have their usual meanings.
\item[] \hbox{\tt amssymbols}
Loads AMS math symbol fonts and associated symbol definitions.
The file {\tt amstable.tex} in the documentation directory
lists the symbols.
\item[] \hbox{\tt apalike}
For use with the \verb|apalike| bibliography style (see below).
\item[] \hbox{\tt cyrillic}
Loads AMS Cyrillic fonts, accessible via \verb|\cyr|.
The file {\tt cyrmemo.tex} in the documentation directory
has transliteration information.
\item[] \hbox{\tt decalign}
Adds `d' for `decimal alignment' as a legal field specifier
in a \verb|tabular| environment.
The column entries should all be of the form $x.y$ where $x$
and $y$ are non-empty.
\item[] \hbox{\tt Draft}
Show overfull boxes with marginal marks, and put the symbolic
names of {\em labels\/} in the margin.
\item[] \hbox{\tt fixup}
Fixes some \LaTeX\ spacing bugs, most importantly in \verb|eqnarray|.
Should come after \verb|fleqn| if that option is used.
\item[] \hbox{\tt fullpage}
Narrow the margins so that most of the page is used.
With 10 point fonts this makes documents harder to read.
\item[] \hbox{\tt nofloats}
Causes floating things (figures, tables) to be thrown
away, while still processing the contents so that figure numbers, etc., get
incremented.  This is useful for papers to be submitted to journals, which
often require figures and tables to be typeset separately.
\item[] \hbox{\tt numinsec}
Used with \verb|siam|.
Make equations, figures, and tables have numbers
the form {\it x.y} where {\it x} is the section number.
\item[] \hbox{\tt proc}
Used with \verb|article| to make two-column output for ACM and
IEEE conference proceedings.
A \verb|\copyrightspace| command should be placed after any \verb|\footnote|'s
in the first column, to make room for the copyright notice.
\item[] \hbox{\tt rmthm}
Redefines the {\tt theorem} and similar environments
so that the text is in roman type instead of the italic that the
standard design styles use.
\item[] \hbox{\tt showidx}
Used with \verb|report| or \verb|book|.
Print marginal notes showing index entries.
(Issues a \verb|\flushbottom|.)
\item[] \hbox{\tt showlabels}
Change \verb|\label|, \verb|\ref|, and \verb|\pageref| to show
symbolic labels with the values.
\item[] \hbox{\tt verbfile}
Defines \verb|\verbatimfile|, a command with one argument---the name
of a file to include in a verbatim environment.
Also \verb|\verbatimlisting|, which prints line numbers too.
\end{itemize}

A few more style options will be discussed in subsequent sections.
 
\section{Bibliographies}

The preferred way to make bibliographies is to use \BibTeX.
At the place where you want the bibliography to appear, put:
\begin{eg}
\bibliography{mybib}
\bibliographystyle{plain}
\end{eg}
Where \verb|mybib.bib| is a bibliography database file and
\verb|plain.bst| is a bibliography style file.
Both should be in one of the directories \LaTeX\ searches for inputs.%
\footnote{Normally the current directory, followed by the standard
macro directory.  You can set the \verb|TEXINPUTS|
shell environment variable to a colon-separated list of directories
to search.}
Then, after \LaTeX\ has run once (to write citation keys into
an \verb|aux| file), run
\begin{eg}
bibtex mypaper
\end{eg}
where \verb|mypaper.tex| is the name of your \LaTeX\ source file.
\LaTeX\ will have to be run twice more before the citations in the
text are finally correct, but then they will continue to be correct
until a change in the works cited or the bibliography database
necessitates rerunning \BibTeX.

You should prepare the database file as described in Appendix B.1
of the \LaTeX\ book, and use one of these bibliography styles:
\begin{itemize}
\item[] \hbox{\tt abbrv}
A version of \verb|plain| using more abbreviations.
\item[] \hbox{\tt acm}
The style used in ACM Transactions.
\item[] \hbox{\tt alpha}
A version of \verb|plain| using alphabetic cite keys
(like ``[Knuth84]'').
\item[] \hbox{\tt apalike}
Something like the APA style,
using ``(Jones, 1986)'' in the text and no label in the reference list.
Not recommended, but required by some journals.
You must use the \verb|apalike| document style option with this.
\item[] \hbox{\tt atttm}
Similar to {\tt unsrt}, but closer to the style
recommended in {\it The Bell Labs Style Guide}.
\item[] \hbox{\tt ieeetr}
IEEE Transactions style: numeric labels, unsorted reference list,
and quotes around article titles.
\item[] \hbox{\tt plain}
The standard, recommended style: numeric labels, reference list
sorted by author, year, title.
\item[] \hbox{\tt siam}
SIAM publication style: like \verb|plain|, but formatted differently
(e.g., names in small caps).
\item[] \hbox{\tt siamu}
Like \verb|siam|, but unsorted reference list.
\item[] \hbox{\tt unsrt}
Like \verb|plain|, but unsorted reference list.
\end{itemize}
You can use the \verb|openbib| document style option with any of
these to get an ``open'' style of bibliography, where there are
several line breaks within entries.

\BibTeX\ has undergone a major revision since the \LaTeX\ book
was written, so some of the information in its Appendix~B.1
is out of date, and there are more features available now.
The new features:
\begin{itemize}
\item
A \verb|\nocite{*}| command will cause a listing of all the entries
in a database file.
\item
A `\verb|#|' character can be used to concatenate strings
in field values.
\item
An entry can assign a cite-key to a field named \verb|crossref|
to cross-reference the other entry (probably a proceedings or
book containing the entry in question).
This allows fields to be inherited, and/or may cause the cross-referenced
work to appear as a separate bibliography entry (to avoid duplicating
information).
Put all cross-referenced entries at the end of a \verb|bib| file.
\item
A \verb|@PREAMBLE{"defs"}| in a database file causes
\verb|defs| to be written before the reference list,
allowing the definition of macros that may be used in titles, etc.
\item
If a name field ends with ``{\tt and others}'', the standard styles
will append ``et al.''
\end{itemize}

In addition to the usual three-letter abbreviations for the months, the
following abbreviations are defined by the bibliography styles:
\begin{list}{}{\leftmargin 0pt
	\labelwidth.7in \leftmargin .8in \itemsep .1in
	\def\makelabel#1{#1\hfill}%
}\it
\item[\tt acmcs] ACM Computing Surveys
\item[\tt acta] Acta Informatica
\item[\tt cacm] Communications of the ACM
\item[\tt ibmjrd] IBM Journal of Research and Development
\item[\tt ibmsj] IBM Systems Journal
\item[\tt ieeese] IEEE Transactions on Software Engineering
\item[\tt ieeetc] IEEE Transactions on Computers
\item[\tt ieeetcad]
 IEEE Transactions on Computer-Aided Design of Integrated Circuits
\item[\tt ipl] Information Processing Letters
\item[\tt jacm] Journal of the ACM
\item[\tt jcss] Journal of Computer and System Sciences
\item[\tt scp] Science of Computer Programming
\item[\tt sicomp] SIAM Journal on Computing
\item[\tt tocs] ACM Transactions on Computer Systems
\item[\tt tods] ACM Transactions on Database Systems
\item[\tt tog] ACM Transactions on Graphics
\item[\tt toms] ACM Transactions on Mathematical Software
\item[\tt toois] ACM Transactions on Office Information Systems
\item[\tt toplas] ACM Transactions on Programming Languages and Systems
\item[\tt tcs] Theoretical Computer Science
\end{list}

There is a document called \verb|btxdoc.tex| in the documentation
directory that has a slightly corrected version of the \LaTeX\ book's
\BibTeX\ documentation, along with some helpful hints for building
databases.
There is also \verb|btxhak.tex|, which explains the \BibTeX\ style file
language, in case you need to design a different style.

\section{Output Devices and Fonts}

\LaTeX\ output is in ``\TeX\ dvi'' format;
it has to be converted by a {\em device driver\/} into the
format required by the destination printer or previewing program.
Normally, {\em lp\/} invokes the correct device driver automatically,
but there are occasions when you need to be aware of this step.
Our local device drivers are {\em dvips}, which converts to PostScript,
and {\em dvit}, which converts to the format used by device-independent
troff.
Our printers are now PostScript printers;
{\em dvit\/} is useful for generating input for {\em proof}.

There is a \TeX\ font directory%
\footnote{Usually \verb|/usr/lib/tex/fonts|, but the
shell environment variable \verb|TEXFONTS| can be set to change or add to it.}
that contains information about fonts that can be used in \LaTeX.
All fonts need a {\em tfm file}, containing character dimensions and
certain other information, so you can see what fonts are available
by doing
\begin{eg}
ls *.tfm
\end{eg}
in the font directory.

The {\em dvips\/} device driver can handle two types of fonts:
built-in and bitmap.
Built-in fonts, like {\tt Times-Roman}, are known to the PostScript printer
and can be used at any magnification (unmagnified usually corresponds
to 10 point size).
PostScript names are often too long for our UNIX
file system, so \LaTeX\ has to use abbreviations.
The \verb|psconfig| file in the miscellaneous \TeX\ directory%
\footnote{Usually \verb|/usr/lib/tex|.}
lists the built-in PostScript fonts and their abbreviations.
By convention, they all begin with capital letters and contain hyphens.
If use use the \verb|-t| or \verb|-p| flag to {\em latex},
you get a version that uses the built-in Times Roman and Palatino families,
respectively, instead of Knuth's Computer Modern fonts.

Bitmap fonts, like {\tt cmr10}, are those for which the PostScript file
needs bitmaps for each character used (once per size at which it is used).
The bitmaps are kept in {\em pk files\/} in the \TeX\ font directory.
The name of a pk file is \mbox{{\em font}.{\em v\/}{\tt pk}}, where
\[
v={\rm round}(\hbox{\it dots per inch}\times\hbox{\it mag factor})
\]
The Laserwriters have a 300 dot per inch resolution, \verb|cmr10.300|pk
is the 10 point bitmap file for Computer Modern Roman 10pt,
and \verb|cmr10.432pk| is a 14.4 point version.%
\footnote{Standard \TeX\ font magnifications are $\sqrt{1.2}$, $1.2$, $1.2^2$,
$1.2^3$, $\ldots$}
The computer center's high resolution phototypesetter has a 1270 dot per
inch resolution.
The bitmaps for those fonts are so large that they may not exist on
your machine.
You should probably run
\begin{eg}
dvips -Tlino paper.dvi > paper.t
\end{eg}
before sending something to the phototypesetter, to see if {\em dvips\/}
complains about missing bitmap fonts.
We can make fonts at that resolution if we have Metafont source
programs for them;  unfortunately, there are no Metafont
source programs for the Cyrillic and AMS symbol fonts.

\section{Graphics}

There are a number of possibilities for getting graphics into
\LaTeX\ documents.
Most of them require special capabilities from
the device driver.
Unfortunately, there is no universal standard for such capabilities,
so if you need to create source documents that will be portable to
other sites you should probably stick to the commands of the
\verb|picture| environment, described in
Section~5.5 of the \LaTeX\ book.
No special capabilities are needed since that environment
uses ordinary characters to draw graphics,
but there is the corresponding disadvantage that you are
restricted to certain line angles and circle diameters.

Our local device drivers incorporate the ability to respond to the so-called
``{\em tpic\/} specials'', a semi-standard set of commands that can
be put in the output file using the \TeX\ \verb|\special| command.
The following subsections describe several ways of doing graphics
using the {\em tpic\/} specials.

\subsection{{\tt picture} Environment Extensions}

The \verb|picext| document style option defines more commands that can
be used in a \verb|picture| environment, and redefines some commands
to be more general:
\begin{itemize}
\item[] \hbox{\verb|\circle{r}|}
	like \LaTeX's \verb|\circle|, but works for any radius.
\item[] \hbox{\verb|\ellipse{a}{b}|}
	draw an ellipse with horizontal radius $a$ and vertical radius $b$.
\item[] \hbox{\verb|\ray(dx,dy)|}
	like \LaTeX's \verb|\line|, but works for any angle.
\item[] \hbox{\verb|\pline(x1,y1)(x2,y2)...(xn,yn)|}
	draw line segments from $(x1,y1)$ to $(x2,y2)$, etc.
\item[] \hbox{\verb|\pvector(x1,y1)(x2,y2)|}
	like a two-point \verb|\pline|, but also draw an arrowhead at $(x2,y2)$.
\item[] \hbox{\verb|\pspline(x1,y1)(x2,y2)...(xn,yn)|}
	draw a quadratic B-spline with the given control points
	(the endpoints are doubled, so the spline goes through them).
\item[] \hbox{\verb|\arc{r}{sa}{ea}|}
	draw a circular arc with the circle's center at the current
	point, radius $r$, and going from angle
	$sa$ counterclockwise to angle $se$ (in radians).
\end{itemize}

The \verb|bezier| document style option allows quadratic Bezier
splines to be drawn.
Let $P_{i}$ be the point
with coordinates $(x_{i},y_{i})$, for $i=1$, 2, and 3.  The command
\begin{itemize} \tt
\item[]
\verb|\bezier{|$n$\verb|}(|$x_{1}$,$y_{1}$)($x_{2}$,$y_{2}$)($x_{3}$,$y_{3}$)
\end{itemize}
draws $n$ points on the quadratic Bezier spline determined by the three
points $P_{1}$, $P_{2}$, and $P_{3}$.  The locus of points on this
spline is a parabolic arc from $P_{1}$ to $P_{3}$ having the line
$P_{1}P_{2}$ tangent to it at $P_{1}$ and the line $P_{2}P_{3}$ tangent
to it at $P_{3}$.  Note that $P_{2}$ is {\em not\/} on this arc unless
$P_{1}$, $P_{2}$, and $P_{3}$ are collinear.
This command does the splines using little dots, so it is slow and
takes a lot of memory.

\subsection{Tpic and Ideal}

The troff filters {\em pic\/} and {\em ideal\/} have
been adapted to work with \LaTeX.
The \LaTeX\ version of {\em pic\/} is called {\em tpic},
and the \LaTeX\ version of {\em ideal\/} is invoked by
using a \verb|-tex| flag.
They work best if you put each figure in a separate file
(including the surrounding \verb|.PS|/\verb|.PE| or \verb|.IS|/\verb|.IE|).
Then use the filter to create a \verb|.tex| file, e.g.:
\begin{eg}
tpic fig.p > fig.tex
\end{eg}
and put something like the following in the \LaTeX\ document:
\begin{eg}
\input{fig}
\begin{center}
\mbox{\box\graph}
\end{center}
\end{eg}
As you can see, the filters arrange for your picture to be in
a box called \verb|\graph|.
Any text strings in the figure will be formatted by \LaTeX,
using \LaTeX\ fonts.

\subsection{PostScript}

PostScript printers can do a lot more than the {\em tpic\/} specials allow---you
can shade regions, rotate text, and other interesting things.
The {\em dvips\/} driver allows the inclusion of a arbitrary
PostScript file with a \LaTeX\ command like
\begin{eg}
\vspace{3in}
\special{psfile=myfig.ps}
\end{eg}
Any drawing commands in the \verb|myfig.ps| will draw on the output
page, with the default PostScript coordinate system with an origin
at the point where the \verb|\special| is issued.
You can transform the coordinate system by supplying a list of
comma-separated ${\it key}={\it value}$ items after the \verb|psfile|
assignment.
Allowed keys are: \verb|hoffset|, \verb|voffset|, \verb|hscale|,
\verb|vscale|, and \verb|rotate|.
If supplied, these values are supplied to PostScript {\em translate},
{\em scale}, and {\em rotate} commands, in that order.
Note that you have to know how much vertical space to leave for the figure.

A fancier mechanism for including PostScript figures is the
{\em Psfig\/} package.  To use it, include
\begin{eg}
\input{psfig}
\end{eg}
at the beginning of your document.
Figures to be included should conform to the PostScript
{\em bounding box convention}, which means there should be
a line like
\[
\hbox{\verb|%% BoundingBox:| \em llx lly urx ury}
\]
where the four arguments are the coordinates of the lower left
and upper right coordinates of a rectangle containing the
whole figure, relative to the default PostScript transformation matrix.
You include figures in the \LaTeX\ document by
\begin{eg}
\psfig{figure=myfig.ps}
\end{eg}
which will put the top of the figure at the current position
on the page, and move the current position to just below the figure.
Adding the following after the filename
\begin{eg}
,hsize=2in,vsize=3in
\end{eg}
causes the picture to be scaled to fit those dimensions.
Another option to the \verb|\psfig| macro is \verb|prolog=|{\em file},
causing a prolog of PostScript definitions to be included before
the figure.

Unfortunately, there is no easy way to access the \TeX\ bitmap fonts
in an included PostScript figure.

\subsection{Other Packages}

There are a number of other packages in use in the \LaTeX\ community.
They aren't being used here, but could be supported if there is
a demand.

\begin{itemize}
\item[] \hbox{\tt EPIC, EEPIC}
Similar to {\tt picext} in extending the \LaTeX\ picture environment.
EPIC requires no special capabilities from the device driver,
while EEPIC uses tpic specials.
See \verb|picman.tex| in the documentation directory.
\item[] \hbox{\tt PiC\TeX}
Allows diagrams to be drawn using lots of tiny dots---slow and
memory consuming.
\item[] \hbox{\tt Fig}
Fig is a graphics format.
It can be produced by a graphics editor that runs under X11 or SunView;
it can also be generated from pic code.
Programs in the {\em transfig\/} package translate
the Fig language into PostScript, PIC\TeX, (E)EPIC,
\LaTeX\ \verb|picture| commands, or {\em pic\/}.
\end{itemize}

\section{Acknowledgments}

The {\tt tm} document style comes from Terry Anderson,
and the {\tt acmtrams} document style is by Leslie Lamport.
John Hobby wrote the {\tt fixup} style option,
Charles Karney wrote the {\tt acmsymbols} and {\tt cyrillic}
style options,
and Leslie Lamport wrote the {\tt alltt}, {\tt proc}, and {\tt showidx}
style options.
The rest of the document styles and options described here
were written by me;
Eric Grosse and Bill Coughran helped refine the {\tt siam}
and {\tt attletter} styles.
The bibliography styles were written by me, and then
rewritten by Oren Patashnik.
The Psfig package comes from Trevor Darell.
Sunil Podar and Conrad Kwok did EPIC and EEPIC,
and the Fig software was developed by various people,
including Supoj Sutanthavibul, Micah Beck, and Ken Yap.

\begin{thebibliography}{9}
\bibitem[1]{latex} Lamport, L. {\em \LaTeX: A Document Preparation System}.
	Addison-Wesley, Reading, MA, 1985.
\end{thebibliography}

\newpage
\null
\vspace{3in}
\Large divider
\vspace{1in}
\Large Programming

\newpage
\null

\end{document}
