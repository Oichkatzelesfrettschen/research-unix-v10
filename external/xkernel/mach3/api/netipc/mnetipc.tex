\documentstyle{article}

\begin{document}
\large{ROUGH DRAFT of MachNetIPC internals documentation}

\section{Overview}
This documentation is an informal introduction to the organization
of the MachNetIPC source code.  It is an evolving document.

MachNetIPC is a complex xkernel protocol.  It is divided into two
major modules: the message translation and port management.  Port
management is organized as a ``co-protocol'' with MachNetIPC.  The
two protocols communicate via a set of interface calls.  The port
transfer protocols, srx and rrx, communicate with port management
to transfers rights involving 3 or more hosts.

\section{Major modules}

\begin{description}

\item {nns.c, nns_boot.c, nns_procs.c}
This module provides an entry point to MachNetIPC using the netname 
interface. A registry mapping service is implemented that can be 
accessed using the mig generated netname interface routines: 
netname_check_in, netname_check_out and netname_look_up.
The SSR protocol is used for remote host registrations. Broadcast name
lookup requests are handled using the UDP protocol.

\item {machnetipc.c}
The heart of the system, for processing the mach\_msg\_header and its
network representation, and containing the ``fast-path'' code, is
contained in this file.  The remote and local port are translate to
and from network form, and rpc's are recognized.  If the message is
``complex'' as determined by the msgh bits, then the message sections
are interpreted by {\em mach-msg-intr}.

Messages enter this module from one of three places:
\begin{itemize}
\item The SSR (Simple ServeR) protocol.	 This forwards server requests
to remote machines.

\item From a thread dedicated to ``listening'' on a Mach port.  Such a
thread is started whenever a network send right is received from a
remote peer.  The routine ``machr\_msg\_receive'' is the main loop of such
a thread.

\item From the network via ``machr\_calldemux.''  There are six types
of messages recognized:

\begin{itemize}
\item SSR messages.  These are client requests for send rights to servers.

\item MACHRIPC messages.  These are forwarded Mach messages requiring no
reply.

\item RPC requests.  These are Mach messages with a limited structure to
facilitate ``fast-path'' processing.  The reply port is a send\_once right,

\item RPC replies.  These are replies to RPC requests.  They are returned
on the same xkernel session on which the request was made.  The lower-level
transport protocol assures the association.  The message is not complex.

\item PORTMGMT messages.  These are housekeeping messages used by the
port management module.  Technically, they should be sent and received
by the port management protocol itself, but as there are only two
simple messages, they have been included in the MachNetIPC module;
they are conveyed to the port management module via the portm\_mgmt
subroutine call.
\end{itemize}
\end{itemize}

\item {port-maint.c}
All port manipulations are carried out in this module.  This includes
converting to and from network form, synchronization, local database
updates, etc.  The port transfer protocols are invoked from here
when needed.


\item {machmsg_int.c}
All port manipulations are carried out in this module.  This includes
all processing of the local or network forms, exclusive of the
mach\_msg\_hdr.

\item {mach-xfer.c}
Some routines that are directly called by the port transfer protocols
are here.  They port transfer protocols use the network port data
structure ``mnetport'' for this.

\item {ssr.c}
The simple server protocol, a simple implementation of a subset of
the netnameserver functionality.  It supports the nns interface.

\item {hdr-utils.c}
The low-level utilties for creating and interpreting network headers
are intended to reside here.  The source machine type is indicated by
a tag that determines the byte swap order.  Header load and store
routines should call these routines if swapping is required.  Routines
and tags for handling floating point numbers are here.

\end{description}

Entry points to machnetipc.c:
\begin{description}
\item {machr\_init}
The initialization of the machnetipc subsystem is done here.  Called
by the xkernel init function, this routine calls on the port manager
module to fork itself into a separate protocol.  Machnetipc configures
itself to provide a push and calldemux interface.

\item {machr\_push}
A highly specialized interface is provided by machr\_push to the upper
protocol ``ssr'' (simple server).  This allows the simple server to
send hybrid messages to other nodes; the hybrid message is addressed
to an IP address, and it contains a server id number, a request type,
and a Mach port right for sending a reply.  This allows nodes to
``bootstrap'' their communication with servers on remote machines.  No
reply is expected through this interface; the Mach reply port will be
used instead, and MachNetIPC will manage communication over that port.

\item {machr\_calldemux}
All incoming network traffic goes through this routine.  There are
three main types of communication: the ssr bootstrap messages, normal
Mach IPC traffic in network format, and port management messages.
Most calls to this routine will result in an immediate zero-length
reply.  For Mach IPC RPC-style traffic, there usualy will be a reply message.
For normal Mach IPC traffic, the calldemux thread will deliver the
Mach message by converting the message from network to local form,
releasing the xkernel master lock, and doing a mach\_msg send operation,

\item {machr\_msg\_receive}
Local Mach IPC traffic enters through a thread running this
routine.  When an incoming port send right is ready for delivery via
mach\_msg, a listening thread is established for the send right, and
it runs this routine.  These listening threads have the port state
bound into them.  The threads are somewhat analogous to xkernel
sessions.

\end{description}
	
Entry points to port-maint.c:
\begin{description}
\item {convert\_to\_netport}
A local Mach port right is converted to its network representation here.
The port database for the port reflects its global network number, the
receiver node, and the make sender count.  The port is locked during
message transit to assure that the destination never receives ``stale''
port information.

\item {convert\_netport\_to\_mach\_port}
A network port descriptor is expanded to a full database descriptor in
this routine.  If the incoming descriptor designates a new send right,
a listening thread is started.  The port transfer protocols are called
from here if the port has three or more machines participating in a
conversation.

\item {convert\_netport\_to\_tmp\_mach\_port}
When a new port right is tranferred using the port transfer protocols, this
routine is called to enter the information into the port database.
The port right is not fully instantiated until a Mach message containing
a reference to the port is received.

\item {port\_mgmt}
Network notifications regarding no more senders and port death are
handled here.

\item {quick\_port\_lookup}
Maps a local Mach port name to its associated database structure.

\item {quick\_netport\_lookup}
Maps a global port name to its associated database structure.

\item {portm\_add\_sender}
Used to add send right reference to port.  This is used for reply ports.
It is a quick version of convert\_to\_netport.

\item {portm\_fast\_send\_once}
The server side of an RPC gets a cached port to use in generating
a send once right for a reply.

\item {portm\_use\_fast\_send\_once}
Returns the server side reply right to the cache.

\item {portm\_get\_netnum}
The client side of an RPC uses this to assign a global port number to
a request.  The server IP address is recorded to allow graceful
deletion of the reply right if the server reboots.

\item {portm\_use\_netnum}
The client side of an RPC removes the record of the server address.

\item {portm\_port\_remove\_send\_once}
Older form of portm\_use\_netnum?

\item {portm\_delete\_local\_receive}
Called when a task holding a network receive right deallocates the right.

\item {portm\_delete\_local\_send}
Called when a task holding a network send right deallocates the right.
May be redundant.

\end{description}

Entry points to machmsg_intr.c:
\begin{description}
\item {xkmach\_msg\_convert}
Convert a Mach message to network form and do all necessary port bookkeeping.
A list of locked embedded ports is attached to the resulting xkernel message
as an attribute.  These ports should be unlocked after the message is sent.

\item {ports\_and\_ool\_data\_convert}
Convert a network message into a Mach message appropriate for local delivery.
Embedded port rights are reconstituted into local form, out-of-line data
is identified, etc.
\end{description}

Entry points to mach-xfer.c:
\begin{description}
\item {addNewSender}
Add a sending node to the list of remote senders for a Mach port.

\end{description}

\end{document}
