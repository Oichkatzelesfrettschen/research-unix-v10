%     
% $RCSfile: mnipc-install.tex,v $
%
% x-kernel v3.2
%
% Copyright (c) 1993,1991,1990  Arizona Board of Regents
%
%
% $Log: mnipc-install.tex,v $
% Revision 1.3  1993/12/01  23:00:09  menze
% msg_internal.h added to list of file modifications
%
% Revision 1.2  1993/11/30  17:19:24  menze
% Rewrote section on specially-compiling msg.c, etc.
%
% Minor fixes to conform to the rest of the manual's typographical
% conventions
%



\section{Installing MachNetIPC - A Network Extension of Mach3 IPC}
\label{installmni}

By compiling a protocol graph with the machnetipc protocol
and its supporting protocols, one can construct a server that
runs in the Mach3 environment and turns the native IPC mechanism into
one that extends across a set of machines.
This can be used as replacement for the Mach program known as
the ``netmsgserver.''  MachNetIPC is faster than the
netmsgserver and supports all Mach3 IPC semantics and the netname
interface.  In addition, the protocols can work over internetworked
configurations.  This release is a beta test release.  We
expect and welcome problem reports (see section \ref{mnibugs}).  Our testing
has established that the transmission of simple messages appears
correct and robust; complex messages involving port transfers are
correct and robust in so far as our environment has allowed testing.
Our test platforms have been HP 720's running OSF/1 from Utah,
DecStations, running MK74 and UX35 and higher, and Intel 486-based
machines running MK82 and UX41.

The files in the mach3/api/netipc directory comprise the interface
between the user tasks and the protocols that transport data.  When
they are composed with a protocol graph providing the transport
and liveness services they require,
the resulting {\sanss xkernel} image file can run as a task that supports
the netnameserver interface for registering and retrieving Mach ports
on other machines running the same software.  NB this software will not
interoperate with the netmsgserver distributed by CMU.

The implementation of MachNetIPC is described in the Usenix Proceedings of
the Mach III Symposium, Santa Fe NM, April 1993.


\subsection{Building MachNetIPC}

To build a MachNetIPC service, begin by creating a ``build'' directory
as described in \ref{building}.  Copy the file 
{\sanss mach3/api/netipc/graph.comp}
to that directory and configure the graph.comp file to refer to the
name of the ethernet device for your system.

You can then type {\tt make compose; make depend; make} to build an
xkernel image file.  

\subsection{Running MachNetIPC}

Application programs registering and using cross-machine services make
use of the netname interface to MachNetIPC.  The ``nns'' interface to
MachNetIPC can operate either as the sole name server on a node, or in
cooperation with a service that is already running (such as ``snames''
or ``netmsgserver''), although this latter configuration slightly
complicates the initial contact with the service.  See the appendix
section about ``nns'' for more information on how to use this service.
The host administrator should decide whether to run MachNetIPC alone
or in cooperation with other services.

The MachNetIPC service can be added to the boot time server startup
script {\sanss nanny.config} in place of the ``netmsgserver''.  Consult
your local system administration guide for more information about
server startup procedures.  Note that the $x$\/-kernel must be started
with a user id that has permission to access the master device port.
If your site does not a RARP server (an unusual circumstance), you may
find it necessary to create a ROM file and start the xkernel image file
from the directory with the rom file.  See section \ref{romfile} and
the ARP man page in appendix \ref{protman} for details.

If the $x$\/-kernel terminates or is terminated, it can be restarted
easily (``xkernel \&'').

\subsection{Using MachNetIPC}

You can use the xkernel image built with MachNetIPC in place of the older
netmsgserver.  The most common way of utilizing it takes advantage of
the netname interface, which is supported by the netmsgserver and by
the ``snames'' program.  This supports registering a name string and a
port for access by remote machines, and lookup of the port
corresponding to the name.  The ports can be used just like any other
Mach port port (with respect to send and receive operations).

Documentation for the netname interface can be found in either the
/usr/mach/man area on your system or in this $x$\/-kernel distribution
in the {\sanss mach3/api/netipc/netname\_*.3} files.

CAVEAT: This release does support using DNS hostnames for netname lookups.
Use the full IP address of remote hosts.  See also the protocol
appendix section for the ``nns'' service.

There is an aspect of the netname interface that depends on having
a broadcast address available.  If one machine wants to look up
a name without specifying the IP address of the destination, a
broadcast query is used.  If the machine offering the name is not
accessible via broadcast, then the lookup will fail.  See section
\ref{mniptbl} for information about setting the UDP port number for
this service.

With the exception of broadcast name lookup, all MachNetIPC functions
work transparently between machines with local area network or
Internet connectivity.

\subsection{Protocol Tables}
\label{mniptbl}

The protocol table defines the UDP port number to use for broadcasts.
We have set it to be the same as is used by the netmsgserver.  If this
causes problems at your site, you can change the entry at the nns
line:

\begin{quote}
\begin{verbatim}
udp     5
{
        pmap    111
        nns     x7823
}
\end{verbatim}
\end{quote}

The illustration shows that the hex number ``7823'' is used.

If your site will not be using the broadcast mechanism for netname
lookup, the protocol number is irrelevant.


\subsection{Large Messages}

There is no inherent limit on the total size of a message sent using
Mach IPC.  MachNetIPC has a default configuration that is limited
to sending datagrams no larger than 64K.  Programs that transfer large
amounts of out-of-line data may be inconvenienced by this.  We provide
an alternate configuration that supports arbitrarily large messages.
We also note that very large message transfers may adversely affect
the performance of a host node, due to having to buffer the message
in the $x$\/-kernel task while it is being transferred (this is worse for
the receiver than the sender).

The file {\sanss mach3/api/netipc/graph.comp.large} can be used in 
place of the
standard graph.comp to support large messages.  It relies on
a newly developed protocol (RAT) that implements a reliable datagram
protocol over TCP.

If you wish to 
use RAT for very large messages (more than 1M byte or so), you will
need to recompile several \xk{} files in a special manner.  We
recommend that you recompile these files in your local build
directory in order to avoid conflicts with other users.  The relevant
files are:

\begin{quote}
\begin{verbatim}
xkernel/pi/msg.c
xkernel/pi/msg_internal.h
xkernel/mach3/pxk/malloc.c
xkernel/mach3/pxk/stack.c
\end{verbatim}
\end{quote}

\noindent
The file {\sanss msg.c} contains the message tool implementation,
which must be compiled with the MSG\_NON\_REC flag to work with large
messages.  The standard Mach malloc doesn't return and reuse large
storage blocks very well, so we recommend compiling our distributed
version of {\sanss malloc.c} for large messages.  
You might also wish to compile and use the {\sanss stack.c}
that we provide; it limits cthread stacks to 64K and
uses less virtual memory.

To compile special versions of these files:

\begin{itemize}

\item{}
Copy (or link) these files to your build directory.  Edit your
Makefile in your build directory to indicate that these files will be
locally compiled, by adding to the PRIVSRC and PRIVOBJ variables:

\begin{quote}
\begin{verbatim}
PRIVSRC = \
        msg.c    \
        malloc.c \
        stack.c  

PRIVOBJ = \
        ./$(HOW)/msg.o    \
        ./$(HOW)/malloc.o \
        ./$(HOW)/stack.o
\end{verbatim}
\end{quote}

\item{}
Add -DMSG\_NON\_REC to the TMP\_CPPFLAGS variable in your Makefile.

\item{}
Run {\tt make depend} and {\tt make}.  

\end{itemize}



\subsection{Access Control and Privacy}

This release of MachNetIPC has no access control or privacy
mechanisms.  We have expectations of providing these services in 1994.


\subsection{Reporting Problems}
\label{mnibugs}
Problems can be reported to xkernel-bugs@cs.arizona.edu, but problems
and discussion that are specific to MachNetIPC should go to
xkernel-mnipc@cs.arizona.edu; if you wish to be a member of this list
send a note to xkernel-mnipc-request.  If you aren't sure which list
to use for a problem, use xkernel-bugs.
