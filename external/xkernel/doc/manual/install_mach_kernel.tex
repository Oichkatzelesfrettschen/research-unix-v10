%
% installing_mach_kernel.tex
%
% $Revision: 1.7 $
% $Date: 1993/11/15 21:50:22 $
%

\section{Installing within a Mach3 Kernel}
\label{installingmachk}

The \xk{} can be compiled as an integral part of the Mach3 kernel.
This facility is provided for researchers developing highly efficient
protocols and for obtaining timing results that are exclusive of
Mach IPC and Unix server overhead.  At the present time, there is no
user interface for accessing the \xk{} services -- the Mach3 kernel
can boot, run an \xk{} timing test and print the results.  A more
usable interface will be provided in the future.

These are instructions for building a version of the \xk{} that resides
in the Mach3 kernel.  You should be familiar with the build instructions
for the ``out of kernel'' version before starting.  You should also be
familiar with the procedure for building the Mach kernel and have a
working environment set up.

The files distributed (and these directions) have been used with the
MK82 Mach kernel for a DecStation 5000 and an Intel 486 PC.


Set your path to include the /usr/xkernel/bin and /usr/xkernel/bin/ARCH
directories.


\begin{enumerate}

\item Place the \xk{} source tree in the mach source tree.
In the src/mk/kernel directory, in either a shadow area or the primary
source area, create a link named 'xkern' to the head of the \xk{}
source tree.  


\item Patch some mach kernel source files.
The tree rooted at xkernel/mach3/machkernel is a parallel of the mach
kernel source tree, containing modifications to support the \xk{}.
These modifications are all conditional on \xk{} compilation flags, so
you can use these modified files to build non-\xk{} Mach kernels if
you desire.  You can either go through these files one at a time and
make the appropriate file-by-file substitutions, or you can set up a
mach shadow directory and link/copy the directories in
xkernel/mach3/machkernel to the shadow area.  The latter should be the
easier approach (and less intrusive to your original mach source
tree.)

Note that one can not transparently shadow the mk/kernel/conf
directory.  Since the directory\\ xkernel/mach3/machkernel/conf 
only includes files
that we have modified, it should be overlayed on top of a copy of an
original mk/kernel/conf directory.

\item Compose an \xk{} graph.
Create a directory xkern.local under src/mk/kernel. This directory will
contain the files created by composing the graph.comp. 
Change to the new xkern.local directory. Copy or create your
graph.comp here. To compose this graph.comp run:

\begin{quote}
\begin{tt}
   	compose -f < graph.comp
\end{tt}
\end{quote}

This must be explicitly done each time the graph.comp changes. The files
traceLevels.c, protocols.c, protocols.h and protTbl.c will be created.
The files that were created here are referenced by the `conf/files' file for
inclusion in the mach kernel.


\item 
In your xkern.local directory, build a compilable version of the
protocol table by running the \xk{} utility program ptbldump:
 
\begin{quote}
\begin{tt}
  	ptbldump /usr/xkernel/etc/prottbl > ptblData.c
\end{tt}
\end{quote}

This file will be automatically included by an \xk{} source file.


\item 
In your xkern.local directory, create a compilable rom file by running
the genrom.awk script:

\begin{quote}
\begin{tt}
   	awk -f /usr/x32/merge/bin/genrom.awk < rom > initRom.c
\end{tt}
\end{quote}

This file will be automatically included by an \xk{} source file.


\item Build a configuration including the \xk{}.  If your
configuration includes the string {\tt +xk}, the \xk{} will be built
in DEBUG configuration (with tracing enabled.)  If the configuration
includes the string {\tt +xko}, the \xk{} will be built in OPTIMIZE
configuration (with no tracing.)  


\item If you have problems with kernels crashing because of storage exhaustion,
   you might want to adjust the zone block table in the Mach kernel file
   kalloc.c.  See the \xk{} file xkernel/pi/msg\_internal.h for the
   size of a message block; these account for most of the \xk{} storage use.


\item The machkernel will come up by default with both the Unix server
and the \xk{} receiving and sending network packets (incoming
packets are copied to both the \xk{} and the Unix server.)  For
accurate kernel-kernel timings of \xk{} protocols, boot the
mach\_kernel image (which doesn't start the user task) instead of the
mach.boot image and configure the in-kernel driver (xklance) to not
deliver incoming packets to the user task as described in Appendix
\ref{protman}. 


\end{enumerate}
