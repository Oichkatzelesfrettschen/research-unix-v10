% 
% $RCSfile: utilities.tex,v $
%
% x-kernel v3.2
%
% Copyright (c) 1993,1991,1990  Arizona Board of Regents
%
%
% $Log: utilities.tex,v $
% Revision 1.6  1993/11/30  17:29:03  menze
% Added sections on "Host Name Service" and "ROM file parsing utilities"
%


\section{Utility Routines}

\subsection{Storage: xMalloc and xFree}

These are just like Unix malloc and free functions.  The {\em xMalloc}
function will cause an $x$-kernel abort if no storage is available;
therefore, it has no error return value.
\medskip

{\sem char} {\bold *xMalloc}({\sem int} {\caps size})\\
\medskip

{\sem int} {\bold xFree}({\sem char} *{\caps buf})

\subsection{Time}

The \xk{} uses the same time structure as Unix:

\begin{tabbing}
xxxx \= xxxxxxxx \= xxxxxxxxxxx \= \kill
\>{\sem typedef struct} \{\\
\>\>  {\sem long}  sec;\\
\>\>  {\sem long}  usec;\\
\>\} {\bold XTime} ;\\
\end{tabbing}


\subsubsection{xGetTime}

Sets {\em *t} to the current time of day:
\medskip

{\sem void} {\bold xGetTime} ({\sem XTime} *{\caps t})


\subsubsection{xAddTime}

Sets the {\em res} to the sum of {\em t1} and {\em t2.}
Assumes {\em t1} and {\em t2} are in standard time format (i.e., does not
check for integer overflow of the usec value.)
\medskip

{\sem void} {\bold xAddTime} 
({\sem XTime} *{\caps res}, {\sem XTime} {\caps t1}, {\sem XTime} {\caps t2})


\subsubsection{xSubTime}

Sets the {\em res} to the difference of {\em t1} and {\em t2.}
The resulting value may be negative.
\medskip

{\sem void} {\bold xSubTime} 
({\sem XTime} *{\caps res}, {\sem XTime} {\caps t1}, {\sem XTime} {\caps t2})
\medskip




\subsection{Byte Order: ntohs, ntohl, htons, and htonl}

The byte order functions are the same as the Unix functions.
\medskip

\begin{tabbing}
xxxx \= xxxxxxxx \= xxxxxxxxxxx \= \kill
\>{\sem u\_short} {\bold ntohs}({\sem u\_short} {\caps n})\\
\>{\sem u\_long} {\bold ntohl}({\sem u\_long} {\caps n})\\
\>{\sem u\_short} {\bold htons}({\sem u\_short} {\caps n})\\
\>{\sem u\_long} {\bold htonl}({\sem u\_long} {\caps n})\\
\end{tabbing}

\subsection{Checksum}

\subsubsection{inCkSum}

Calculates a 16-bit 1's complement checksum over the buffer
{\em *buf} (of length {\em len}) and the msg {\em *m}, returning the bit
complement of the sum.  {\em len} should be even and the buffer must be
aligned on a 16-bit boundary.  {\em len} may be zero.
\medskip

{\sem u\_short} {\bold inCkSum} ({\sem Msg} *{\caps m}, {\sem u\_short} *{\caps buf}, {\sem int} {\caps len});

\subsubsection{ocsum}

Returns the 1's complement sum of the {\em count} 16-bit words pointed
to by {\em hdr}, which must be aligned on a 16-bit boundary.
\medskip

{\sem u\_short} {\bold ocsum} ( {\sem u\_short} *{\caps hdr}, {\sem int} {\caps count});


\subsection{Strings to Hosts}

Utility routines exist for converting from string representations of
IP and Ethernet addresses to their structural counterparts and
vice-versa.  

\subsubsection{ipHostStr}

Returns a string with a ``dotted-decimal'' representation of IP host 
{\em h} (e.g., ``192.12.69.1'')  
This string is in a static buffer, so it must be copied if
its value is to be preserved.
\medskip

{\sem char *} {\bold ipHostStr} ( {\sem IPhost} *{\caps h} )


\subsubsection{str2ipHost}

Interprets {\em str} as a ``dotted-decimal'' 
representation of an IP host and assigns the fields of 
{\em h} accordingly.  The operation fails if {\em str} does not seem
to be in dotted-decimal form.
\medskip

{\sem xkern\_return\_t} {\bold str2ipHost} 
( {\sem IPhost} *{\caps h}, {\sem char } *{\caps str} )


\subsubsection{ethHostStr}

Returns a string with a representation of Ethernet host 
{\em h} (e.g., ``8:0:2b:ef:23:11'')  
This string is in a static buffer, so it must be copied if
its value is to be preserved.
\medskip

{\sem char *} {\bold ethHostStr} ( {\sem ETHhost} *{\caps h} )


\subsubsection{str2ethHost}

Interprets {\em str} as a six-hex-digit-colon-separated 
representation of an Ethernet host and assigns the fields of 
{\em h} accordingly.  The operation fails if {\em str} does not seem
to be in the correct format.
\medskip

{\sem xkern\_return\_t} {\bold str2ethHost} 
( {\sem ETHhost} *{\caps h}, {\sem char } *{\caps str} )


\subsection{Host Name Service}
\label{dns}
A simple way of mapping host name strings to host IP addresses is
provided via rom file entries (see section \ref{romfile}) and the interface
function {\em xk\_gethostbyname}.

During $x$-kernel startup, rom file lines beginning with
the string ``dns'' are parsed into name and address components
and added to the host name table.  E.g., 

\begin{verbatim}
dns umbra 192.12.69.97
\end{verbatim}

The host name must be less than 64 characters in length.

\subsubsection{xk\_gethostbyname}

This function will look up a hostname and return its IP address in
storage provided with the call.
The name must be an exact match to a rom file entry, NB no substrings
are allowed.  If the name is not found, the return code indicates
failure.
\medskip

{\sem xkern\_return\_t} {\bold xk\_gethostbyname} ({\sem char} *{\caps name}, 
{\sem IPhost} *{\caps addr} );



\subsection{ROM file parsing utilities}

When writing a protocol that provides user-configurable ROM file
options, one can make use of the ROM file parsing utilities to process
the ROM file entries.  To use these utilities:

\begin{enumerate}

\item{}
Write separate routines to handle each ROM option your
protocol will support.  These routines should be of the following
type: 

\begin{tabbing}
xxx \= xxxxxxxxxx \=   \kill
\>{\sem typedef xkern\_return\_t} ({\bold *XObjRomOptFunc} ) \\
\>\>(
{\sem XObj} {\caps protl}, 
{\sem char} **{\caps fields}, 
{\sem int} {\caps numFields}, 
{\sem int} {\caps lineNumber}, 
{\sem void} *{\caps arg} 
)
\end{tabbing}

The ROM file parsing code will call this handler when it finds an
appropriate line in the ROM file.  The number of fields on that line
and the fields themselves will be placed in {\caps numFields} and
{\caps fields}, respectively.  The line number is
provided to allow the handler to produce error messages if desired.

If the handler returns XK\_FAILURE, the parsing code will print a
generic ``ROM file format error'' message, specifying the name of the
protocol and the line number.  


\item{}
Create an array of XObjRomOpt structures which bind 
option names to their handling functions.  There is one XObjRomOpt
structure for each ROM option.

\begin{tabbing}
xxx \= xxxxxx \= xxxxxxxxxxxxxxxxx \= xxxxxxxxx \=  \kill
\>{\sem typedef struct} \{\\
\>\>{\sem char} \>*name; 
        \>{\smallfont /* name of the option as specified in ROM file */}\\
\>\>{\sem int}	 \>minFields;  
        \>{\smallfont /* Minimum number of fields for this option */}\\
\>\>{\sem XObjRomOptFunc}	\>func;       
        \>{\smallfont /* Handler function */}\\
\>\} {\bold XObjRomOpt};

\end{tabbing}


\item{}
Somewhere in your protocol's initialization code, call findXObjRomOpts.

\hfil{\sem xkern\_return\_t}
{\bold findXObjRomOpts}(
{\sem XObj}, {\sem XObjRomOpt} *, {\sem int} {\caps numOpts}, 
{\sem void } *{\caps arg}
);

This routine scans through the ROM file, looking for lines where the
first field matches either the protocol name or the
full instance name of the XObject (e.g., if the protocol instance is
{\tt ethdrv/SE0}, ROM file entries with either {\tt ethdrv/SE0} or 
{\tt ethdrv}
would match.)  When such a match is found, the array of XObjRomOpts is
scanned.  If the second field of the line matches the name field of
one of the XObjRomOpts, or if the name field of one of the XObjRomOpts is
the empty string, the XObjRomOptFunc for that option is called with
the object, all ROM fields on that line, the number of fields on that
line, the line number and the user-supplied argument.

If the first field of a ROM line appears to match the XObject but none
of the supplied RomOpts matches the second field, an error message
will be printed and XK\_FAILURE will be returned.  The rest of the ROM
entries will not be scanned.  This same behavior results from the
XObjRomOptFunc returning XK\_FAILURE and from a ROM line with too few
fields for its associated handler.

\end{enumerate}


As an example, consider a protocol which supports two ROM options, a
{\em port} option and an {\em mtu} option, both of which take single integer
parameters.  This example shows how this protocol
would interface with the romopt parsing utilities:

\break

\begin{verbatim}

static xkern_return_t	readMtu( XObj, char **, int, int, void * );
static xkern_return_t	readPort( XObj, char **, int, int, void * );


static XObjRomOpt opts[] = {
    { "mtu", 3, readMtu },
    { "port", 3, readPort }
};


static xkern_return_t 
readMtu( XObj self, char **arr, int nFields, int line, void *arg )
{
    PState	*ps = (PState *)self->state;

    return sscanf (arr[2], "%d", &ps->mtu) < 1  ? XK_FAILURE : XK_SUCCESS;
}

static xkern_return_t 
readPort( XObj self, char **arr, int nFields, int line, void *arg )
{
    PState	*ps = (PState *)self->state;

    return sscanf (arr[2], "%d", &ps->port) < 1  ? XK_FAILURE : XK_SUCCESS;
}



foo_init( XObj self ) 
{
    ...
    findXObjRomOpts(self, opts, sizeof(opts)/sizeof(XObjRomOpt), 0);
    ...
}


\end{verbatim}
