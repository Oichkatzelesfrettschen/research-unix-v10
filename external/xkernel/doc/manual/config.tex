% 
% $RCSfile: config.tex,v $
%
% x-kernel v3.2
%
% Copyright (c) 1993,1991,1990  Arizona Board of Regents
%
%
% $Log: config.tex,v $
% Revision 1.24  1993/11/30  17:08:27  menze
% Added intelx86 to list of supported architectures
% Noted change in gnumake version
%
%

\section{Configuring a Kernel}
\label{config}

This section describes how to configure and build an \xk{}.

\subsection{Specifying a Protocol Graph}
\label{compose}

A new instance of the $x$-kernel is configured by specifying the
collection of protocols that are to be included in the kernel in a
file called {\sanss graph.comp}. Example {\sanss graph.comp} files can
be found in {\sanss xkernel/build/Template}.

The {\sanss graph.comp} file is divided into three sections: device
drivers, protocols, and miscellaneous configuration parameters.  The
sections are separated by lines beginning with {\sanss @;} and each
section may be empty.

The first two sections, device drivers and protocols, describe the
protocol graph to be configured into your kernel.  The only difference
between the two sections is that drivers in the first section are
initialized directly from the \xk{} boot thread, whereas protocols
in the protocol section are initialized from a distinct protocol
initialization thread.  For the device drivers and platforms in this
distribution, this distinction is of no consequence and device drivers
may be configured in either the first or the second section.

Device drivers and
protocols are described by the same types of entries, as in the
example:

\let\tt=\COURIERtt
\begin{verbatim}
name=yap files=yap,yap_clnt,yap_srvr dir=yap protocols=ip,eth trace=TR_MAJOR_EVENTS;
\end{verbatim}
\let\tt=\CMRtt

\noindent
The first field gives the protocol's name.  The rest of the fields are
optional and may occur in any order.  The {\sanss dir} and {\sanss
files} fields describe the names and locations of the source files
used to build the protocol.  Files are specified without extensions.
These two fields should not be used in the common case where one wants
to link in protocol object code from the public system object area.
The {\sanss dir} and {\sanss files} fields should only be used when
one wants to compile and link code from one's private build
area instead.  If a {\sanss files} entry exists but no {\sanss dir}
entry is specified, the current directory (i.e., the build directory)
is assumed.  If a {\sanss dir} entry exists without a {\sanss files}
entry, the {\sanss files} field defaults to a single {\sanss .c} file
with the protocol's name.

The {\sanss protocols} field indicates the protocols below the current
protocol in the graph.  When this field contains multiple protocols, order is
significant (the lower protocols will be loaded into the upper
protocol's down vector in the order in which they are listed.)
A protocol which expects multiple protocols below it will describe the
expected semantics of the lower protocols in its manual page in
appendix \ref{protman}.

The {\sanss trace} field defines the debugging level used in trace
statements depending on the protocol variable {\sanss traceyapp}.  See
section \ref{kdebug} for the symbolic names associated with trace levels.

Multiple instantiations of protocols are supported by using a ``/''
character after the protocol name to add an extension.  In the
following example, two instantiations of ``yap'' are indicated, one
over ``ip'' and one over ``eth,'' and both are used by the ``prt''
protocol.  In this example, the instance suffixes for the ``yap''
protocol are the names of the protocol below the instance, but this is
just a convention -- they could have any distinct string as an
instance suffix.
\let\tt=\COURIERtt
\begin{verbatim}
name=yap/ip files=yap,yap_clnt,yap_srvr dir=yap protocols=ip trace=TR_MAJOR_EVENTS;
name=yap/eth protocols=eth;
name=prt files=prt dir=prt protocols=yap/ip,yap/eth trace=TR_ERRORS;
\end{verbatim}
\let\tt=\CMRtt

\noindent Only the first of multiple instantiations should have {\sanss dir},
{\sanss files}, or {\sanss trace} fields.

The third section of graph.comp contains the names of files with
protocol number tables that are to be loaded during initialization.
It also contains the names of subsystems and their configuration
parameters.  Currently trace variables are the only parameters that
can be set here.  The following illustrates a typical use of the third
section:

\let\tt=\COURIERtt
\begin{verbatim}
@;
#
# You can also specify protocol tables to be read in at boot time.
#
prottbl=/usr/xkernel/etc/prottbl;
prottbl=./prottbl.local;

#subsystem tracing for messages and protocol operations
name=msg	trace=TR_GROSS_EVENTS;
name=protocol	trace=TR_MAJOR_EVENTS;
\end{verbatim}
\let\tt=\CMRtt

The graph.comp file is read by an $x$-kernel utility program called
{\em compose} which generates startup code to build the protocol
graph and set up the described configuration.  The protocol graph is
built bottom-up; when a protocol's initialization function is called,
the lower level-protocols have already been initialized.

\subsection{Build Procedure}
\label{building}

A given instance of the \xk{} is built in a working directory.  Each
working directory can support one \xk{} configuration at a time.
Working directories may be organized within the \xk{} source tree
(e.g., as subdirectories of {\sanss xkernel/build}) or outside of
the source tree (e.g., in an \xk{} user's home directory.)  

For the purpose of the following discussion, we assume the user is
configuring a kernel so as to implement and evaluate protocol YAP 
(yet another protocol.) A
user configuring a kernel that contains only existing protocols should
ignore all references to YAP.  The machine architecture in this
example is ``sparc'' (used by the sunos platform); other supported 
architectures are ``mips'' (for the DECstation 5000), and ``intelx86''.

\begin{enumerate}

\item{}
Put {\sanss xkernel/bin/sparc} and {\sanss xkernel/bin} in
your search path. These must be before
{\sanss /bin} and {\sanss /usr/bin}. This will allow use of
the version of {\sanss make} distributed with the $x$-kernel (GNU make
v. 3.66) rather than the standard Unix {\sanss make}.

\item{}
Create a new working directory to hold your kernel.


\item{}
Change to your new working directory.

\item{}
Copy the Makefile
from {\sanss xkernel/build/Template/Makefile.sunos} into your
working directory under the name Makefile:

\begin{quote}
{\sanss cp xkernel/build/Template/Makefile.sunos Makefile}\\
\end{quote}

\item{}
Make the Makefile writable:
\begin{quote}
{\sanss chmod u+w Makefile}\\
\end{quote}

\item{}
Edit the Makefile.
Select an appropriate {\sanss HOWTOCOMPILE}. Generally, you want {\sanss
DEBUG} until you are ready to run serious performance tests, then
change it to {\sanss OPTIMIZE}. (See Section \ref{kdebug}.)

Check to see that the variable XRT in your Makefile is a path to
the root of the xkernel source tree.


\item{}
If you are including a new protocol in your kernel, then create (at least)
a shell for that protocol; e.g., create a shell version of {\sanss yap.c}.
The protocol shell should define an initialization routine; 
e.g., {\sanss yap\_init}.  You will also need to put an entry for your
protocol in one of the protocol tables referenced by your graph.comp
(see section \ref{protnum}.)
If you are configuring a kernel that contains only existing protocols, 
then this step is not necessary.

\item{}
Create a {\sanss graph.comp} in your build directory, as discussed in
section \ref{compose}, to specify the protocol graph for your kernel.
Example graph.comps are located in the {\sanss xkernel/build/Template}
directory.

\smallskip

In steps \ref{en:make_compose}, \ref{en:make_depend}, and \ref{en:make_system},
you may see one or more compiler warning messages of the
form:

\begin{verbatim}
         make[n]: fopen: [filename]: No such file or directory
\end{verbatim}

These messages are expected and can be ignored.
The files referenced by these messages will be created in
steps \ref{en:make_depend} and \ref{en:make_system}.

The following
two steps must be redone if {\sanss graph.comp} is later modified:
\item{}
\label{en:make_compose}
{\sanss make compose}

\item{}
\label{en:make_depend}
{\sanss make depend} 

\item{}
\label{en:make_system}
If you are the first person building a kernel of this configuration at
your site (i.e., if you are building a kernel as part of the
installation procedure), you will need to run an initialization
target for the system areas:
\begin{quote}
{\sanss make system}
\end{quote}
This will run the targets {\em allCompose}, {\em allDepend}, and {\em
libs} for the configuration and platform indicated by your Makefile,
building configuration files, dependency files, and object files for
most of the \xk{} system.  Again, you will get error messages about
missing dependency files, but the {\em allDepend} target will generate
the missing files. 
This will take a long time to run.  The
installer of the \xk{} should run this target for all configurations
(DEBUG,OPTIMIZE) and platforms that will be used at the site.

\item{}
{\sanss make}
\\Object files will be placed in a subdirectory whose name reflects your
configuration and platform (DEBUGsunos, OPTIMIZEmach3, etc.)  This is
the same way in which object files are stored throughout the \xk{}
hierarchy.  The final \xk{} executable ({\tt xkernel}) will be placed
in your build directory. 

\end{enumerate}

\subsection{Debug versus Optimized Mode}

The $x$-kernel uses a trace package to generate debugging information.
To enable the tracing facility, edit the Makefile to set {\sanss
HOWTOCOMPILE} to {\sanss DEBUG}; tracing can be disabled by setting {\sanss
HOWTOCOMPILE} to {\sanss OPTIMIZE}. Then type {\sanss make}.  See also
section \ref{kdebug}.

If you are interested in accurate performance timings, you should set
{\sanss HOWTOCOMPILE} to {\sanss OPTIMIZE} in the Makefile.  This
causes all trace of tracing code to be eliminated from the kernel.



\subsection{Useful Make Targets}

The previously mentioned make targets ({\em compose}, {\em depend},
{\em xkernel} (the default)) are the minimal set necessary to build an
\xk{}.  This section describes some additional useful make targets.

Some of these descriptions refer to ``\xk{} system areas''.  These
areas contain source and object code shared by all users of the
\xk{}.  Each platform has four main system areas:


\begin{quote}
	prot:
\begin{quote}
		Code for the \xk{} protocols.
		({\tt /usr/xkernel/protocols})
\end{quote}
\smallskip
	
	pi:
\begin{quote}
		Platform-independent code common to all
		platforms.  ({\tt /usr/xkernel/pi})
\end{quote}
\smallskip

	pxk:
\begin{quote}
		Platform-specific code including device
		drivers and application programmer interfaces (APIs).
		(e.g., {\tt /usr/xkernel/mach3},
		{\tt /usr/xkernel/sunos}, etc.)
\end{quote}
\smallskip

	user:
\begin{quote}
		Platform-specific user libraries that can be linked
		with user applications to access the \xk{} through
		one of the \xk{} APIs.  (e.g., {\tt
		/usr/xkernel/mach3/user})  Not all platforms have such
		libraries. 
\end{quote}
\smallskip
\end{quote}
		

% \subsubsection{Make Targets}
 
These targets are all executed from a build directory.

\begin{quote}
prot, pi, pxk, user, libs:
\begin{quote}
	Rebuilds any out-of-date source code in the indicated area.
	The target {\em libs} rebuilds code in the {\em prot},
	{\em pi}, and {\em pxk} areas. 
\smallskip
\end{quote}

clean:
\begin{quote}
	removes object files in the build directory.
\smallskip
\end{quote}

protClean, piClean, pxkClean, userClean, allClean:
\begin{quote}
	Removes object files, libraries and other generated files from
	the indicated area.  The {\em allClean} target runs in all areas
	except the user area.
\smallskip
\end{quote}

protCompose, piCompose, pxkCompose, allCompose:
\begin{quote}
	Rebuilds ComposeFiles lists for the indicated area.  These
	files group object files together so the compose program 
	knows which files to link in for which protocols.
	These files are generated by the {\em system} make target
	during system installation, but they
	are only updated when these targets are run explicitly. 
	The {\em allCompose} target runs in all areas except the user
	area. 
\smallskip
\end{quote}

protDepend, piDepend, pxkDepend, userDepend, allDepend:
\begin{quote}
	Rebuilds the dependency lists in the indicated area.
	The {\em allDepend} target runs in all areas except the user area. 
\smallskip
\end{quote}


srcList, hdrList:
\begin{quote}
	Builds files in the local build directory listing all of
	the active \xk{} source files.  These targets will do
	nothing if the files already exist.  To force the files to be
	rebuilt, use the targets {\em SrcList} and {\em HdrList}
	instead. 
\smallskip
\end{quote}

grep:
\begin{quote}
	greps for the target G in all \xk{} source and header 
	files produced by the {\em srcList} and {\em hdrList} targets. 

	Example usage:

\begin{quote}
		{\tt make grep G="xCreate"}
\smallskip
\end{quote}
\end{quote}

tags:
\begin{quote}
	Creates an emacs tag file {\tt TAGS} in the local build directory 
	for all \xk{} source and header
	files produced by the {\em srcList} and {\em hdrList} targets. 
\smallskip
\end{quote}

\end{quote}


\subsection{Modifying Code in System Areas}

In order to make the common case as fast as possible, the \xk{}
build procedure normally doesn't check to see if any source code in
the system areas needs to be recompiled.  If you make a change to any
of the system areas, that code will have to be recompiled explicitly
from the build area using the targets for that area.  For example, if
you add a protocol to the system protocol area, in addition to
modifying the protocol Makefile you should run these targets:

\begin{verbatim}
        make protDepend	
        make protCompose	
        make prot
\end{verbatim}

