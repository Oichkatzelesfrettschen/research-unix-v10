%
% $RCSfile: nns.tex,v $
%
% x-kernel v3.2
%
% Copyright (c) 1993,1991,1990  Arizona Board of Regents
%
%
% $Revision: 1.5 $
% $Date: 1993/11/29 21:23:17 $
%

\subsection*{NAME}

\noindent NNS (Network Name Service)


\subsection*{SYNOPSIS}

NNS provides an entry point to MachNetIPC using the netname
interface. A registry mapping service can be 
accessed using the MIG generated netname interface routines: 
netname\_check\_in, netname\_check\_out and netname\_look\_up.
Documentation for the netname interface can be found either in
/usr/mach/man or in the xkernel distribution in 
{\sanss mach3/api/netipc/netname\_*.3}
manpages. NNS will not interfere with a previously executing name
server. To guarantee that NNS is being used during checkins, 
lookups and checkouts, do a lookup on the service ``NetworkNameServer''
and use the associated port as the default name server port. That way
requests will not mistakenly be directed to another name server instead.
See mach3/user/netipc/test/nnstest.c for more details.

\subsection*{REALM}

NNS is in the anchor protocol realm; it is specific to the Mach3 platform.

\subsection*{CONFIGURATION}

Broadcast name lookup requests are handled using the UDP protocol.
NNS depends on the BIDCTL protocol to detect reboots. Additionally the SSR
protocol must be configured below NNS. See notes below for details of their 
interface.

\noindent {\tt name=nns protocols=udp, bidctl;}

\subsection*{NOTES}

NNS utilizes the SSR protocol for bootstrapping remote host ports into
MachNetIPC. NNS and SSR communicate using a globally defined Mach port which 
requires that they be configured into the same Mach task.

\subsection*{CAVEATS and DISCLAIMERS}

To specify a host in the remote lookup, use the IP address of the host
and not the host name. Host name resolutions are currently not supported.

Only a single instance of NNS may run at any time.

\subsection*{AUTHOR}

\noindent Hasnain Karampurwala

