
% rat.tex
%
% $Revision: 1.2 $
% $Date: 1993/07/20 21:11:03 $
%

\subsection*{NAME}
\label{RAT}

\noindent RAT ( Record Accessed TCP )

\subsection*{SPECIFICATION}

\noindent
RAT uses TCP to ship very large packets to remote hosts. 

\subsection*{SYNOPSIS}
RAT is a protocol uses TCP to send very large packets to remote hosts. The 
current implementation of RAT manages a single TCP connection between 
each host pair. Messages pushed to RAT are in turn pushed to TCP. Unlike 
TCP RAT preserves packet boundaries. One the receiving end RAT 
reconstructs the original packet from an arbitrary list of buffers handed 
to it by TCP. RAT should be able to accept packets as large as the message 
tool implementation supports. 

Note that RAT is totally responsible for the management of the TCP 
connection. In the current implementation this connection is not opened 
until a message is pushed to RAT. In the future more than one TCP 
connections between any two hosts could be supported without affecting 
the protocols which use RAT. 

Currently  a compiled constant in RAT determines the TCP send and 
receive buffer sizes. 

RAT will almost always perform worse than BLAST on packets that are 
small enough for BLAST to deliver. 

RAT uses any message attribute attached to the outgoing message 
to communicate the fact that it is reliable to some higher level protocol
(in this case CHAN). 

\noindent 
\subsection*{REALM}

RAT is in the ASYNC realm.

\subsection*{PARTICIPANTS}

The interface to RAT is intended to look as much like IP, VNET, and BLAST 
as possible. RAT assumes that the first participant passed to it at open 
time contains the IP address of the remote host. RAT openEnable takes 
participant which is ignored. OpenEables are based upon the protocol ID of 
the higher level protocol.

\subsection*{CONTROL OPERATIONS}

RAT passes all control operations to IP the second protocol configured. 

\subsection*{CONFIGURATION}

RAT is not freely composable. It must be configured as follows:

name=rat protocols=tcp, ip

RAT has been designed to work under VSIZE with either BLAST, VNET, or IP as the short message protocol. 
\
bigskip

\noindent
RAT recognizes the following ROM options:

\smallskip

{\tt rat/xxx sourcePort N}: Instantiation {\em xxx} of RAT 
will use port N as its TCP source port. If N is equal to zero
TCP will chose a port at random.

{\tt rat/xxx destPort N}: Instantiation {\em xxx} of RAT 
will use port N as its TCP destination port. Each instance of 
RAT must have its own destination port. 

\medskip

\subsection*{AUTHORS}

\noindent David C. Schwartz and Sean W. O'Malley



