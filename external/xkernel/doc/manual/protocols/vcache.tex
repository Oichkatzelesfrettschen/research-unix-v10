\subsection*{NAME}

\noindent VCACHE ( Virtual Caching Protocol )


\subsection*{SPECIFICATION}

VCACHE is a simple session caching protocol which provides a time
buffer between when an upper protocol closes a lower session and when
that close is propagated to the lower session.


\subsection*{SYNOPSIS}

When a VCACHE session is closed, it is placed in a time-expiration
cache.  When VCACHE receives subsequent opens (active or passive), it
first looks in its cache for sessions with the same hlp and the same
remote host.  After a configurable amount of time, closed VCACHE
sessions will be destroyed and the lower session actually closed.

As an example, VCACHE can be inserted between a UDP server and UDP to
prevent UDP sessions from being closed when the server releases all
references.

Server sessions (those created as a result of a match between an
incoming packet and an openenable) are always cached.  If an
instantiation of VCACHE is running in {\em symmetric} mode, client
sessions (those created by an xOpen) will be cached as well.  It is
reasonable to configure VCACHE in symmetric mode above symmetric
(ASYNC realm) protocols and in asymmetric mode above asymmetric (RPC
realm) protocols, though this usage is certainly not required.


\subsection*{REALM}

VCACHE can be used with either ASYNC or RPC realm lower protocols and
assumes the realm of the protocol below it.


\subsection*{PARTICIPANTS}

VCACHE passes participants directly to the lower protocol and does not
make direct use of any participant information.



\subsection*{CONFIGURATION}

VCACHE uses a single lower protocol.

\noindent
VCACHE recognizes the following ROM options:

\begin{description}

\item[{\tt vcache symmetric}]:
\smallskip
Run this instantiation in symmetric mode.

\item[{\tt vcache asymmetric}]:
Run this instantiation in asymmetric mode (default).

\item[{\tt vcache interval n}]:
Set the timeout interval for {\tt n} seconds.  A cached VCACHE session
is eligible to be destroyed if it is idle for {\tt n} seconds and is
guaranteed to be destroyed if it is idle for more than {\tt 2 * n}
seconds. 

\end{description}

\subsection*{AUTHOR}

\noindent Ed Menze
