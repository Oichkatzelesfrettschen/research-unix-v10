%
% $RCSfile: bidctl.tex,v $
%
% $Revision: 1.4 $
% $Date: 1993/11/18 06:04:55 $
%

\subsection*{NAME}

\noindent BIDCTL (Bootid Control Protocol)

\subsection*{SPECIFICATION}

BIDCTL is the control module of the BootId protocol. 
The BootId protocol is designed to advise workstations that a peer has
rebooted, to protect protocols from receiving messages generated
during previous boot incarnations, and to inform higher protocols of a
peer's reboot in a timely fashion.

If an upper protocol registers with BIDCTL protocol and messages
from its session pass through BID sessions, the BootId protocol
guarantees that a message from a rebooted peer will not be sent to an
upper protocol until the upper protocol has been informed of the
reboot.  

\subsection*{SYNOPSIS}

\noindent 
Upper protocols register their desire to be informed of a peer's
reboot by openEnabling BIDCTL with that remote peer's IPhost.  When
BIDCTL determines that the remote peer has rebooted, it informs all
interested upper protocols via a control operation (see below.) If an
upper protocol is no longer interested in learning about a peer's
reboot, it may openDisable BIDCTL.

\subsection*{REALM}

BIDCTL is in the CONTROL realm.  There are no BIDCTL sessions.

\subsection*{PARTICIPANTS}

BIDCTL openEnable and openDisable expect a single participant
containing the IPhost of the remote peer.

\subsection*{CONTROL OPERATIONS}

\begin{description}

\item[{\tt BIDCTL\_PEER\_REBOOTED:}]
Invoked {\em by} BIDCTL on registered upper protocols to inform them
that a peer has rebooted.  The {\em id} field of the BidctlBootMsg
contains the new remote BootId.  The upper protocol's control function
should not block while handling this notification.

\begin{description}
\item[{\rm Input:}] { \tt BidctlBootMsg ==  \{ IPhost h; BootId id; \} }
\item[{\rm Output:}] none
\end{description}

\end{description}

The remaining control operations are not necessary for most users of
BIDCTL.  They are provided mostly for the use of filtering protocols
(e.g., BID)
which work in conjunction with BIDCTL.

\begin{description}

\item[{\tt BIDCTL\_FIRST\_CONTACT:}]
Invoked {\em by} BIDCTL on registered upper protocols to inform them
that an initial bootid for the given peer has been discovered.  
The {\em id} field of the BidctlBootMsg
contains the new remote BootId. 
\begin{description}
\item[{\rm Input:}] { \tt BidctlBootMsg }
\item[{\rm Output:}] none
\end{description}

\item[{\tt BIDCTL\_GET\_LOCAL\_BID:}]
Returns the current BootId of the local host.  
\begin{description}
\item[{\rm Input:}] none
\item[{\rm Output:}] {\tt BootId }
\end{description}

\item[{\tt BIDCTL\_GET\_PEER\_BID:}]
Returns the last confirmed BootId of the given remote host.  
If the {\em
id} field of the input structure is non-zero, it indicates a possible
new value.  If this value differs from BIDCTL's confirmed value for
that peer,
BIDCTL will start a handshake with the remote peer to determine a new confirmed
value.  The input id can be used by a filtering protocol to indicate that it
has seen a new remote BootId value.

The BootId of the output structure will be 0
(an invalid BootId) if
BIDCTL doesn't yet know the peer's BootId.
\begin{description}
\item[{\rm Input:}] {\tt BidctlBootMsg }
\item[{\rm Output:}] {\tt BidctlBootMsg }
\end{description}

\item[{\tt BIDCTL\_GET\_PEER\_BID\_BLOCKING:}]
Differs from {\tt BIDCTL\_GET\_PEER\_BID} in that the calling thread
will block if BIDCTL has not yet learned the peer's BootId or if the
suggested {\em id} field is non-zero and differs from the protocol's
current value for the peer BootId.  If the operation blocks, it will
not release the calling thread until the peer BootId has been
confirmed.  There is no timeout.

\begin{description}
\item[{\rm Input:}] {\tt BidctlBootMsg }
\item[{\rm Output:}] {\tt BidctlBootMsg }
\end{description}



\end{description}

\subsection*{CONFIGURATION}

BIDCTL expects only its transport protocol below it.  It will open the
transport protocol with a single participant consisting of the remote
IP host. 

BIDCTL uses an internal checksum and works correctly in the presence
of dropped messages, so a reliable transport protocol is not
necessary. 

As an optimization, BIDCTL can perform an IP local-net broadcast to
inform interested peers that it has rebooted.  A rom file entry of the
form:
\begin{verbatim}
         bidctl	bcast
\end{verbatim}
will cause the broadcast and an entry of the form:
\begin{verbatim}
         bidctl	nobcast
\end{verbatim}
will suppress it.  Without a rom file entry, BIDCTL will perform the broadcast
unless BIDCTL\_NO\_BOOT\_BCAST is defined during
compilation. 


\subsection*{AUTHOR}

\noindent Ed Menze

