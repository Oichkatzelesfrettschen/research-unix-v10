
% JOIN.tex
%
% $Revision: 1.2 $
% $Date: 1993/06/25 23:15:00 $
%

\subsection*{NAME}
\label{JOIN}

\noindent JOIN 

\subsection*{SPECIFICATION}

\subsection*{SYNOPSIS}
\noindent

JOIN merges up to N outgoing messages pushed to N different 
sessions into a single outgoing message. When this message 
arrives it its broken up into its original configuration and demuxed 
to N potentially different protocols. JOIN is bidirectional and 
will perform the same action on the reply. N can be any number 
between 1 and 16. The default value for N is four and other 
values may be set using ROM file options. 

The number of xOpen's and xOpenenable's that should be performed 
on JOIN  and the number of xOpenDone's JOIN will perform on 
higher level protocols should be equal to N + 1. The first open 
will return, and the first open done will produce, a control 
session which is not used to transmit data. To transmit a message 
to user of JOIN would push a single message to one or more of 
the remaining N sessions and the perform and JOINDONE done on 
the control session. When JOINDONE is called on JOIN the protocol 
gathers all of the packets pushed and joins them into a single 
packet and transmits them. 

When the packet arrives at the receiver it is broken up into its 
original configuration and demuxed to the appropriate higher 
level protocols.  If no packet has been pushed to a specific JOIN 
session no packet will be demuxed at the corresponding join 
session at the receiver. When this is finished JOIN will perform 
an upward JOINDONE control operation on the session or protocol 
passed to JOIN by the receiving protocol using the JOINSETCONTROL 
operation.  This allows the user to detect if any of the incoming 
message fragments has been lost or never sent. NOTE some protocol 
must know that it is using JOIN and must perform JOINSETCONTROL 
immediately after the first openDone performed by JOIN on the 
SESSION passed up by the openDone.

The order in which JOIN will demux the messages to the higher 
level protocol(s) can be set in the ROM file.

\noindent 
\subsection*{REALM}

JOIN is in the ASYNC realm.

\subsection*{PARTICIPANTS}

JOIN takes arbitrary participants. The only restriction is that 
JOIN must be opened N+1 times with each open having exactly 
the same participant list. The best way to achieve these duplicate 
opens is for the higher level protocol to perform one open which 
eventually trickles down to JOIN and then do a get participants 
on the resulting session. This participant list should be used to 
perform the remaining opens. JOIN will create one and only one 
set of JOIN session for each unique address. Attempts to open 
another set of JOIN sessions with the same address will fail. 
Note there is a huge potential race condition if the higher level 
protocol which knows about join is in a different scheduling 
domain from JOIN. 

\subsection*{CONTROL OPERATIONS}

JOINGETNUMSEG: Returns N. 

JOINGETORDER: Returns an array of 16 integer which gives the 
order in which JOIN will demux the messages fragments.  While 
all 16 ints are returned only the first N of them should be 
considered valid. If N=4 then 0 1 2 3 4 is a valid order and 4 3 2 
1 is also a valid order. 

JOINDONE: When called on the JOIN control session it cause join 
to collect all fragments currently available and send a composite 
message. JOIN invokes JOINDONE on the higher level protocol or 
session passed to it by JOINSETCONTROL when it is finished 
demultiplexing message fragments. JOINDONE takes no arguments.

JOINSETCONTROL: Takes a protocol or session as argument. Some 
higher level protocol must invoke this operation on the JOIN 
control session before any packets are delivered. The protocol 
implementor must implement a JOINDONE control operation in 
the control procedure. 

subsection*{CONFIGURATION}

The composability of JOIN is rather complex and not particularly 
well understood. There must be some higher level protocol that 
knows it is being compose on top of JOIN. This protocol will be 
configured on top of N+1 protocols. The first protocol must be 
JOIN. Between this protocol JOIN any number of other protocols 
may be composed on each of the remaining   N paths. For example 
the following is a valid join protocol graph.

\begin{verbatim}
name=auth protocols= join, join,join,crypt/one,join,crypt/two;
name=crypt/one protocols=join;
name=crypt/two protocols=join;
name=join ....
\end{verbatim}

Note that the participant lists that propagate to join through 
each of these paths must be identical. Therefore if one path to 
JOIN modifies the participant list all other paths must also 
perform identical modifications. For situations where the 
incoming and outgoing paths differ it may be possible to use 
standard asymmetric protocol graph techniques to allow the 
incoming and outgoing message to pass through different protocol 
graphs. Note that any protocol composed between AUTH and JOIN 
must xPush zero or one message down for each xPush done to it. 
Protocols which fragment messages should not be composed 
between AUTH and JOIN. 

\subsection*{ROM File}

The ROM file is used to set N and the order in which the  message 
parts should be demuxed to higher level protocols. The following 
is an example ROM file:

\begin{verbatim}
join segments 4
join order 3 2 1 0
\end{verbatim}

After the user has set the number of segments to be joined he 
should set the order at which he would like the message segments 
to be demuxed. In this case JOIN will demux the messages in 
reverse order. Note the default N is 4 and the default order is (0 
1 2 3 4 5 6 7 8 9 10 11 12 13 14 15). However the user should 
always set N and the order.

\medskip

\subsection*{AUTHOR}

\noindent Sean W. O'Malley


