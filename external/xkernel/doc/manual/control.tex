%
% $Revision: 1.11 $
% $Date: 1993/11/18 15:51:59 $
%

\section{ Control Operations }
\label{control}

Control operations are used to perform arbitrary operations on
protocols and sessions, via the {\em xControl} operation described in
section \ref{xControl}.  
{\em xControl} returns an integer that
indicates the length in bytes of the information which was written
into the buffer, or -1 to indicate an error.

All implementations of control operations should check the length
field before reading or writing the buffer, returning -1 if the buffer
is too small.  The {\sanss checkLen(actualLength, expectedLength)}
macro can be used for this.

The {\em opcode} field in {\em xControl} specifies the operation to be
performed on the protocol or session.
There are two ``classes'' of operations: standard ones that may be
implemented by more than one protocol, and protocol-specific ones.

\subsection{ Standard Control Operations }
\subsubsection{ Protocol and Session Operations }

These operations can be performed on both sessions and protocols.

\begin{quote}
	GETMYHOST, GETMYHOSTCOUNT
\begin{quote}
When used on a protocol, GETMYHOST asks for all possible host
addresses for the local host.  When used on a session, GETMYHOST asks
for the local host addresses actually being used on the connection.
If the buffer is too small for all of the hosts, GETMYHOST will write
as many hosts as the buffer allows (GETMYHOST with a buffer large
enough to hold one host will return the most common or default host.)
GETMYHOSTCOUNT asks for the number of hosts which will be returned by
GETMYHOST.

\end{quote}

	GETMAXPACKET, GETOPTPACKET
\begin{quote}

Treats the buffer as a pointer to an integer and sets it to the length
of the longest message that the {\em XObj} can deliver (GETMAXPACKET) or
the length of the longest message that can be delivered without
fragmentation (GETOPTPACKET.)  A protocol typically implements this
operation by querying the lower {\em XObj}s and then subtracting its
header length.  

\smallskip
Although GETMAXPACKET and GETOPTPACKET can be performed on protocols,
it is preferable to use them on sessions, since different sessions of
the same protocol may return different values.

\end{quote}

	RESOLVE, RRESOLVE
\begin{quote}

These operations map high-level addresses into low-level
addresses (RESOLVE) and vice versa (RRESOLVE.) 

\end{quote}
\end{quote}

\newpage{}

\subsubsection{ Session Operations }
\begin{quote}
	GETPEERHOST, GETPEERHOSTCOUNT
\begin{quote}
GETPEERHOST returns the host addresses of all peers of a session.  It
is an error to submit a buffer that is too small for all of the peer
hosts, and -1 will be returned.  GETPEERHOSTCOUNT asks for the number
of hosts which will be returned by GETPEERHOST.

\end{quote}
	GETMYPROTO, GETPEERPROTO
\begin{quote}
Treats the buffer as a pointer to a long and sets it to the local or
remote ``protocol number'' of the session.  For example, UDP returns
the local UDP port from a GETMYPROTO operation.

\end{quote}
	GETPARTICIPANTS
\begin{quote}
Treats the buffer as a simple array of bytes.  The lower session
creates a participant list with data that was used to open the
session.  This participant list is then written into the control
buffer using {\em partExternalize} (see section \ref{part}.)  The
caller will have to run {\em partInternalize} on the buffer before it
can be used as a participant list.

Sessions typically implement this control op by first executing it on their
lower session.  If their protocol uses information on participant
stacks, they then run {\em partInternalize} on the buffer, push their
protocol-specific information on the participant stacks, and then
re-externalize the participant list into the buffer.

\end{quote}
	FREERESOURCES
\begin{quote}
Treats the buffer as a pointer to an {\em xmsg\_handle\_t}.  This
value is interpreted as the result of a previous {\em xPush} and frees
the resources associated with that message.

\end{quote}

	SETNONBLOCKINGIO
\begin{quote}
Treats the buffer as a pointer to an {\em int} (non-zero == TRUE.)  
This operation is interpreted by sessions which do output buffering.
Such sessions may block threads executing an xPush until sufficient
buffer space is available to hold the outgoing message.  If
SETNONBLOCKINGIO(true) is performed on such a session, a thread which
would normally block in such a situation returns with an XMSG\_ERR\_WOULDBLOCK message
handle instead.

\end{quote}


\end{quote}

\subsection{ Protocol-Specific Control Operations }

Protocol-specific opcodes are defined relative to an identifier that
has been assigned to each protocol. For example, the protocol ARP
has been assigned the id {\sanss ARP\_CTL}. Individual opcodes are
then defined (in {\sanss arp.h}) as

\begin{tt}\begin{tabbing}
xxxx \= xxxxxxxxxxxxxxxxxxxxxxxxxxxxxxxxxx \= \kill
\> \#define ARP\_INSTALL \> (ARP\_CTL*MAXOPS + 0)\\
\> \#define ARP\_IPINTERFACES \> (ARP\_CTL*MAXOPS + 1)\\
\> \#define ARP\_IPADDRS \> (ARP\_CTL*MAXOPS + 2)
\end{tabbing}\end{tt}

\noindent This scheme is used to ensure that all control opcodes
are unique. By convention, protocol-specific opcodes defined by
protocol XYZ are prefixed with {\sanss XYZ\_}. Also, until an identifier
has been assigned to a protocol you are writing (i.e., {\sanss upi.h} has
been edited), you can use a set of temporary ids: {\sanss TMP0\_CTL},
{\sanss TMP1\_CTL}, etc.

Protocol-specific control operations are described in the manual page
for each protocol in appendix \ref{protman}.

\subsection { Forwarding Control Operations }

There are several situations where an {\em XObj} may not be prepared to
handle a control operation.  For example, a protocol-specific control
operation may be sent through several intermediate {\em XObj}s in a graph
before it reaches an {\em XObj} that understands the operation.  Because
of this, {\em XObj}s should be prepared to forward control operations
which they don't understand or can't satisfy to their lower {\em XObj}s.


