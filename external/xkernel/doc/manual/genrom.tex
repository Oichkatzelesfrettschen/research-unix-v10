% 
% $RCSfile: genrom.tex,v $
%
% x-kernel v3.2
%
% Copyright (c) 1993,1991,1990  Arizona Board of Regents
%
%
% $Log: genrom.tex,v $
% Revision 1.4  1993/11/30  17:12:45  menze
% Became a subsubsection
%


\subsubsection{ROM File Generators}
\label{romgen}

The ``rom'' file used during $x$-kernel initialization has a cryptic
format that can be difficult specify and maintain manually.  The
{\em genrom.sh} and {\em genrom2.sh} utility programs can ameliorate
the problem by using a ``rom'' file template and expanding it into
one more files for use with testing configurations.  This is especially
helpful for use with the {\em simeth} (see appendix \ref{driverman})
and {\em simsimeth}
(see appendix \ref{protman}) protocols, or in testing configurations where an
{\em ARP} server is not available.


The {\em genrom.sh} utility generates rom files for the $x$-kernel from
prototype files.  It includes information about the addresses of hosts
specified by name on the command line.  For example, using the
prototype rom file ``romgen'' with the three lines:

\begin{verbatim}
simeth
arp
dns
\end{verbatim}

\noindent
and the command {\tt genrom.sh romgen umb lei}, two rom files are produced,
rom.umb.1 and rom.lei.2 with the contents

\begin{verbatim}
rom.umb.1:
simeth 4724
arp 192.12.69.97 192.12.69.97 4724
arp 192.12.69.116 192.12.69.116 4725
dns umb 192.12.69.97
dns lei 192.12.69.116

rom.lei.2:
simeth 4725
arp 192.12.69.97 192.12.69.97 4724
arp 192.12.69.116 192.12.69.116 4725
dns umb 192.12.69.97
dns lei 192.12.69.116
\end{verbatim}

The ``dns'' lines refer to the simulated IP addresses for {\em simeth} and
{\em simsimeth} files.  See the manual sections on simsimeth
and simeth for detailed explanations of the simulated address numbers.

The {\em genrom2.sh} utility generates two rom files for the $x$-kernel
from two prototype rom file files.  This is generally used for
developing rom files to be used between a machine running simeth and a
machine running simsimeth.  For example, given two prototype rom
files, romgen1 and romgen2,

\begin{verbatim}
romgen1:
simeth
arp

romgen2:
ethdrv/SE0 priority 200 
arp/lower
eth/upper mtu 1440
simsimeth
arp/upper
\end{verbatim}

The command {\tt genrom2.sh romgen1 romgen2 umbra moz} produces two rom
files, rom.umbra.1 and rom.tch.2 with contents:

\begin{verbatim}
rom.umbra.1:
simeth 1690
arp 192.12.69.97 192.12.69.97 1690
arp 192.12.69.99 192.12.69.99 1691

rom.tch.2:
ethdrv/SE0 priority 200
arp/lower 192.12.69.97 192.12.69.97 1690
arp/lower 192.12.69.99 192.12.69.99 1691
simsimeth 1691
eth/upper mtu 1440
arp/upper 192.12.69.97 c0:0c:45:61:06:9a
arp/upper 192.12.69.99 c0:0c:45:63:06:9b
\end{verbatim}

\subsubsection{CAVEATS}

Both genrom.sh and genrom2.sh work by running the Unix command ``arp''
to get the ethernet addresses of the machines.  The target machines
must be running in order to get this information.  If they aren't
running, a bogus ethernet number will be used as a substitute.  There
is a further complication in that you cannot get the ethernet address
of the machine you are running on.  To overcome this failure, the
programs will try to run ``arp'' on another machine using ``rsh''.
The default machine is ``cheltenham'' (this default should be changed
for sites other than University of Arizona); if you do not have an
account on that machine, you will need to set the environment variable
ROMGEN\_HOST to some other host on your network where you can execute
the command {\tt /usr/etc/arp} via the ``rsh'' command.  It may be
necessary to establish a {\tt .rhosts} file on the ROMGEN\_HOST for
this to work properly.

To get the arp addresses into the routing table, each host is probed
using ``ping''.  Output from ``ping'' indicating whether the host is
dead or alive will be printed on stdout.

It is important that the arp lines come after the simeth or simsimeth
line they are related to.  I.e., in the romgen2 file above, the arp/lower
line generates arp entries appropriate to a real ethernet because there
has been no previous mention of a simeth or simsimeth protocol.
After the ``simsimeth'' line, the next arp line produces entries appropriate
to the simsimeth protocol.

The port numbers used for encapsulating traffic sent over UDP are
assigned at random.  You can override this by assigning a base number
in decimal using the environment variable ROMGEN\_BASE\_PORT.

Normally the simeth arp lines use the same address for the real and
simulated host address.  If you use the same host twice on the command
line (e.g. {\tt genrom.sh romgen umb umb}), then the second arp entry will
use a host address of ``192.12.69.1'', and further redundancies in host
names will simply result in incrementing the fourth component of the
network address.  This might cause clashes with other host names.
You can change the base number used for assigning the fourth component
using the environment variable ROMGEN\_FAKE\_IP\_BASE.

Another problem occurs when one real host is used to simulate more
than one test host.  The ``dns'' lines that are generated must use a
simulated host name.  This is achieved by attaching a number to
hostname, i.e. {\tt genrom romproto umb umb} will result in dns records
for ``umb'' and ``umb2''.  Currently, this is only detected when the
hostname occurs consecutively on the command line, i.e. 

{\tt genrom romproto umb ush umb}

would not be handled correctly; it would be
necessary to edit the generated rom files to correct the dns lines
for the second occurrence of ``umb''.

Only the first word on each line is examined in the prototype file;
if you have ``arp xxx yyyy'' or ``dns name number'' lines, they will
be interpreted exactly as if they were simply ``arp'' or ``dns''.
